
% Default to the notebook output style

    


% Inherit from the specified cell style.




    
\documentclass[11pt]{article}

    
    
    \usepackage[T1]{fontenc}
    % Nicer default font (+ math font) than Computer Modern for most use cases
    \usepackage{mathpazo}

    % Basic figure setup, for now with no caption control since it's done
    % automatically by Pandoc (which extracts ![](path) syntax from Markdown).
    \usepackage{graphicx}
    % We will generate all images so they have a width \maxwidth. This means
    % that they will get their normal width if they fit onto the page, but
    % are scaled down if they would overflow the margins.
    \makeatletter
    \def\maxwidth{\ifdim\Gin@nat@width>\linewidth\linewidth
    \else\Gin@nat@width\fi}
    \makeatother
    \let\Oldincludegraphics\includegraphics
    % Set max figure width to be 80% of text width, for now hardcoded.
    \renewcommand{\includegraphics}[1]{\Oldincludegraphics[width=.8\maxwidth]{#1}}
    % Ensure that by default, figures have no caption (until we provide a
    % proper Figure object with a Caption API and a way to capture that
    % in the conversion process - todo).
    \usepackage{caption}
    \DeclareCaptionLabelFormat{nolabel}{}
    \captionsetup{labelformat=nolabel}

    \usepackage{adjustbox} % Used to constrain images to a maximum size 
    \usepackage{xcolor} % Allow colors to be defined
    \usepackage{enumerate} % Needed for markdown enumerations to work
    \usepackage{geometry} % Used to adjust the document margins
    \usepackage{amsmath} % Equations
    \usepackage{amssymb} % Equations
    \usepackage{textcomp} % defines textquotesingle
    % Hack from http://tex.stackexchange.com/a/47451/13684:
    \AtBeginDocument{%
        \def\PYZsq{\textquotesingle}% Upright quotes in Pygmentized code
    }
    \usepackage{upquote} % Upright quotes for verbatim code
    \usepackage{eurosym} % defines \euro
    \usepackage[mathletters]{ucs} % Extended unicode (utf-8) support
    \usepackage[utf8x]{inputenc} % Allow utf-8 characters in the tex document
    \usepackage{fancyvrb} % verbatim replacement that allows latex
    \usepackage{grffile} % extends the file name processing of package graphics 
                         % to support a larger range 
    % The hyperref package gives us a pdf with properly built
    % internal navigation ('pdf bookmarks' for the table of contents,
    % internal cross-reference links, web links for URLs, etc.)
    \usepackage{hyperref}
    \usepackage{longtable} % longtable support required by pandoc >1.10
    \usepackage{booktabs}  % table support for pandoc > 1.12.2
    \usepackage[inline]{enumitem} % IRkernel/repr support (it uses the enumerate* environment)
    \usepackage[normalem]{ulem} % ulem is needed to support strikethroughs (\sout)
                                % normalem makes italics be italics, not underlines
    

    
    
    % Colors for the hyperref package
    \definecolor{urlcolor}{rgb}{0,.145,.698}
    \definecolor{linkcolor}{rgb}{.71,0.21,0.01}
    \definecolor{citecolor}{rgb}{.12,.54,.11}

    % ANSI colors
    \definecolor{ansi-black}{HTML}{3E424D}
    \definecolor{ansi-black-intense}{HTML}{282C36}
    \definecolor{ansi-red}{HTML}{E75C58}
    \definecolor{ansi-red-intense}{HTML}{B22B31}
    \definecolor{ansi-green}{HTML}{00A250}
    \definecolor{ansi-green-intense}{HTML}{007427}
    \definecolor{ansi-yellow}{HTML}{DDB62B}
    \definecolor{ansi-yellow-intense}{HTML}{B27D12}
    \definecolor{ansi-blue}{HTML}{208FFB}
    \definecolor{ansi-blue-intense}{HTML}{0065CA}
    \definecolor{ansi-magenta}{HTML}{D160C4}
    \definecolor{ansi-magenta-intense}{HTML}{A03196}
    \definecolor{ansi-cyan}{HTML}{60C6C8}
    \definecolor{ansi-cyan-intense}{HTML}{258F8F}
    \definecolor{ansi-white}{HTML}{C5C1B4}
    \definecolor{ansi-white-intense}{HTML}{A1A6B2}

    % commands and environments needed by pandoc snippets
    % extracted from the output of `pandoc -s`
    \providecommand{\tightlist}{%
      \setlength{\itemsep}{0pt}\setlength{\parskip}{0pt}}
    \DefineVerbatimEnvironment{Highlighting}{Verbatim}{commandchars=\\\{\}}
    % Add ',fontsize=\small' for more characters per line
    \newenvironment{Shaded}{}{}
    \newcommand{\KeywordTok}[1]{\textcolor[rgb]{0.00,0.44,0.13}{\textbf{{#1}}}}
    \newcommand{\DataTypeTok}[1]{\textcolor[rgb]{0.56,0.13,0.00}{{#1}}}
    \newcommand{\DecValTok}[1]{\textcolor[rgb]{0.25,0.63,0.44}{{#1}}}
    \newcommand{\BaseNTok}[1]{\textcolor[rgb]{0.25,0.63,0.44}{{#1}}}
    \newcommand{\FloatTok}[1]{\textcolor[rgb]{0.25,0.63,0.44}{{#1}}}
    \newcommand{\CharTok}[1]{\textcolor[rgb]{0.25,0.44,0.63}{{#1}}}
    \newcommand{\StringTok}[1]{\textcolor[rgb]{0.25,0.44,0.63}{{#1}}}
    \newcommand{\CommentTok}[1]{\textcolor[rgb]{0.38,0.63,0.69}{\textit{{#1}}}}
    \newcommand{\OtherTok}[1]{\textcolor[rgb]{0.00,0.44,0.13}{{#1}}}
    \newcommand{\AlertTok}[1]{\textcolor[rgb]{1.00,0.00,0.00}{\textbf{{#1}}}}
    \newcommand{\FunctionTok}[1]{\textcolor[rgb]{0.02,0.16,0.49}{{#1}}}
    \newcommand{\RegionMarkerTok}[1]{{#1}}
    \newcommand{\ErrorTok}[1]{\textcolor[rgb]{1.00,0.00,0.00}{\textbf{{#1}}}}
    \newcommand{\NormalTok}[1]{{#1}}
    
    % Additional commands for more recent versions of Pandoc
    \newcommand{\ConstantTok}[1]{\textcolor[rgb]{0.53,0.00,0.00}{{#1}}}
    \newcommand{\SpecialCharTok}[1]{\textcolor[rgb]{0.25,0.44,0.63}{{#1}}}
    \newcommand{\VerbatimStringTok}[1]{\textcolor[rgb]{0.25,0.44,0.63}{{#1}}}
    \newcommand{\SpecialStringTok}[1]{\textcolor[rgb]{0.73,0.40,0.53}{{#1}}}
    \newcommand{\ImportTok}[1]{{#1}}
    \newcommand{\DocumentationTok}[1]{\textcolor[rgb]{0.73,0.13,0.13}{\textit{{#1}}}}
    \newcommand{\AnnotationTok}[1]{\textcolor[rgb]{0.38,0.63,0.69}{\textbf{\textit{{#1}}}}}
    \newcommand{\CommentVarTok}[1]{\textcolor[rgb]{0.38,0.63,0.69}{\textbf{\textit{{#1}}}}}
    \newcommand{\VariableTok}[1]{\textcolor[rgb]{0.10,0.09,0.49}{{#1}}}
    \newcommand{\ControlFlowTok}[1]{\textcolor[rgb]{0.00,0.44,0.13}{\textbf{{#1}}}}
    \newcommand{\OperatorTok}[1]{\textcolor[rgb]{0.40,0.40,0.40}{{#1}}}
    \newcommand{\BuiltInTok}[1]{{#1}}
    \newcommand{\ExtensionTok}[1]{{#1}}
    \newcommand{\PreprocessorTok}[1]{\textcolor[rgb]{0.74,0.48,0.00}{{#1}}}
    \newcommand{\AttributeTok}[1]{\textcolor[rgb]{0.49,0.56,0.16}{{#1}}}
    \newcommand{\InformationTok}[1]{\textcolor[rgb]{0.38,0.63,0.69}{\textbf{\textit{{#1}}}}}
    \newcommand{\WarningTok}[1]{\textcolor[rgb]{0.38,0.63,0.69}{\textbf{\textit{{#1}}}}}
    
    
    % Define a nice break command that doesn't care if a line doesn't already
    % exist.
    \def\br{\hspace*{\fill} \\* }
    % Math Jax compatability definitions
    \def\gt{>}
    \def\lt{<}
    % Document parameters
    \title{HW1}
    
    
    

    % Pygments definitions
    
\makeatletter
\def\PY@reset{\let\PY@it=\relax \let\PY@bf=\relax%
    \let\PY@ul=\relax \let\PY@tc=\relax%
    \let\PY@bc=\relax \let\PY@ff=\relax}
\def\PY@tok#1{\csname PY@tok@#1\endcsname}
\def\PY@toks#1+{\ifx\relax#1\empty\else%
    \PY@tok{#1}\expandafter\PY@toks\fi}
\def\PY@do#1{\PY@bc{\PY@tc{\PY@ul{%
    \PY@it{\PY@bf{\PY@ff{#1}}}}}}}
\def\PY#1#2{\PY@reset\PY@toks#1+\relax+\PY@do{#2}}

\expandafter\def\csname PY@tok@gd\endcsname{\def\PY@tc##1{\textcolor[rgb]{0.63,0.00,0.00}{##1}}}
\expandafter\def\csname PY@tok@gu\endcsname{\let\PY@bf=\textbf\def\PY@tc##1{\textcolor[rgb]{0.50,0.00,0.50}{##1}}}
\expandafter\def\csname PY@tok@gt\endcsname{\def\PY@tc##1{\textcolor[rgb]{0.00,0.27,0.87}{##1}}}
\expandafter\def\csname PY@tok@gs\endcsname{\let\PY@bf=\textbf}
\expandafter\def\csname PY@tok@gr\endcsname{\def\PY@tc##1{\textcolor[rgb]{1.00,0.00,0.00}{##1}}}
\expandafter\def\csname PY@tok@cm\endcsname{\let\PY@it=\textit\def\PY@tc##1{\textcolor[rgb]{0.25,0.50,0.50}{##1}}}
\expandafter\def\csname PY@tok@vg\endcsname{\def\PY@tc##1{\textcolor[rgb]{0.10,0.09,0.49}{##1}}}
\expandafter\def\csname PY@tok@vi\endcsname{\def\PY@tc##1{\textcolor[rgb]{0.10,0.09,0.49}{##1}}}
\expandafter\def\csname PY@tok@vm\endcsname{\def\PY@tc##1{\textcolor[rgb]{0.10,0.09,0.49}{##1}}}
\expandafter\def\csname PY@tok@mh\endcsname{\def\PY@tc##1{\textcolor[rgb]{0.40,0.40,0.40}{##1}}}
\expandafter\def\csname PY@tok@cs\endcsname{\let\PY@it=\textit\def\PY@tc##1{\textcolor[rgb]{0.25,0.50,0.50}{##1}}}
\expandafter\def\csname PY@tok@ge\endcsname{\let\PY@it=\textit}
\expandafter\def\csname PY@tok@vc\endcsname{\def\PY@tc##1{\textcolor[rgb]{0.10,0.09,0.49}{##1}}}
\expandafter\def\csname PY@tok@il\endcsname{\def\PY@tc##1{\textcolor[rgb]{0.40,0.40,0.40}{##1}}}
\expandafter\def\csname PY@tok@go\endcsname{\def\PY@tc##1{\textcolor[rgb]{0.53,0.53,0.53}{##1}}}
\expandafter\def\csname PY@tok@cp\endcsname{\def\PY@tc##1{\textcolor[rgb]{0.74,0.48,0.00}{##1}}}
\expandafter\def\csname PY@tok@gi\endcsname{\def\PY@tc##1{\textcolor[rgb]{0.00,0.63,0.00}{##1}}}
\expandafter\def\csname PY@tok@gh\endcsname{\let\PY@bf=\textbf\def\PY@tc##1{\textcolor[rgb]{0.00,0.00,0.50}{##1}}}
\expandafter\def\csname PY@tok@ni\endcsname{\let\PY@bf=\textbf\def\PY@tc##1{\textcolor[rgb]{0.60,0.60,0.60}{##1}}}
\expandafter\def\csname PY@tok@nl\endcsname{\def\PY@tc##1{\textcolor[rgb]{0.63,0.63,0.00}{##1}}}
\expandafter\def\csname PY@tok@nn\endcsname{\let\PY@bf=\textbf\def\PY@tc##1{\textcolor[rgb]{0.00,0.00,1.00}{##1}}}
\expandafter\def\csname PY@tok@no\endcsname{\def\PY@tc##1{\textcolor[rgb]{0.53,0.00,0.00}{##1}}}
\expandafter\def\csname PY@tok@na\endcsname{\def\PY@tc##1{\textcolor[rgb]{0.49,0.56,0.16}{##1}}}
\expandafter\def\csname PY@tok@nb\endcsname{\def\PY@tc##1{\textcolor[rgb]{0.00,0.50,0.00}{##1}}}
\expandafter\def\csname PY@tok@nc\endcsname{\let\PY@bf=\textbf\def\PY@tc##1{\textcolor[rgb]{0.00,0.00,1.00}{##1}}}
\expandafter\def\csname PY@tok@nd\endcsname{\def\PY@tc##1{\textcolor[rgb]{0.67,0.13,1.00}{##1}}}
\expandafter\def\csname PY@tok@ne\endcsname{\let\PY@bf=\textbf\def\PY@tc##1{\textcolor[rgb]{0.82,0.25,0.23}{##1}}}
\expandafter\def\csname PY@tok@nf\endcsname{\def\PY@tc##1{\textcolor[rgb]{0.00,0.00,1.00}{##1}}}
\expandafter\def\csname PY@tok@si\endcsname{\let\PY@bf=\textbf\def\PY@tc##1{\textcolor[rgb]{0.73,0.40,0.53}{##1}}}
\expandafter\def\csname PY@tok@s2\endcsname{\def\PY@tc##1{\textcolor[rgb]{0.73,0.13,0.13}{##1}}}
\expandafter\def\csname PY@tok@nt\endcsname{\let\PY@bf=\textbf\def\PY@tc##1{\textcolor[rgb]{0.00,0.50,0.00}{##1}}}
\expandafter\def\csname PY@tok@nv\endcsname{\def\PY@tc##1{\textcolor[rgb]{0.10,0.09,0.49}{##1}}}
\expandafter\def\csname PY@tok@s1\endcsname{\def\PY@tc##1{\textcolor[rgb]{0.73,0.13,0.13}{##1}}}
\expandafter\def\csname PY@tok@dl\endcsname{\def\PY@tc##1{\textcolor[rgb]{0.73,0.13,0.13}{##1}}}
\expandafter\def\csname PY@tok@ch\endcsname{\let\PY@it=\textit\def\PY@tc##1{\textcolor[rgb]{0.25,0.50,0.50}{##1}}}
\expandafter\def\csname PY@tok@m\endcsname{\def\PY@tc##1{\textcolor[rgb]{0.40,0.40,0.40}{##1}}}
\expandafter\def\csname PY@tok@gp\endcsname{\let\PY@bf=\textbf\def\PY@tc##1{\textcolor[rgb]{0.00,0.00,0.50}{##1}}}
\expandafter\def\csname PY@tok@sh\endcsname{\def\PY@tc##1{\textcolor[rgb]{0.73,0.13,0.13}{##1}}}
\expandafter\def\csname PY@tok@ow\endcsname{\let\PY@bf=\textbf\def\PY@tc##1{\textcolor[rgb]{0.67,0.13,1.00}{##1}}}
\expandafter\def\csname PY@tok@sx\endcsname{\def\PY@tc##1{\textcolor[rgb]{0.00,0.50,0.00}{##1}}}
\expandafter\def\csname PY@tok@bp\endcsname{\def\PY@tc##1{\textcolor[rgb]{0.00,0.50,0.00}{##1}}}
\expandafter\def\csname PY@tok@c1\endcsname{\let\PY@it=\textit\def\PY@tc##1{\textcolor[rgb]{0.25,0.50,0.50}{##1}}}
\expandafter\def\csname PY@tok@fm\endcsname{\def\PY@tc##1{\textcolor[rgb]{0.00,0.00,1.00}{##1}}}
\expandafter\def\csname PY@tok@o\endcsname{\def\PY@tc##1{\textcolor[rgb]{0.40,0.40,0.40}{##1}}}
\expandafter\def\csname PY@tok@kc\endcsname{\let\PY@bf=\textbf\def\PY@tc##1{\textcolor[rgb]{0.00,0.50,0.00}{##1}}}
\expandafter\def\csname PY@tok@c\endcsname{\let\PY@it=\textit\def\PY@tc##1{\textcolor[rgb]{0.25,0.50,0.50}{##1}}}
\expandafter\def\csname PY@tok@mf\endcsname{\def\PY@tc##1{\textcolor[rgb]{0.40,0.40,0.40}{##1}}}
\expandafter\def\csname PY@tok@err\endcsname{\def\PY@bc##1{\setlength{\fboxsep}{0pt}\fcolorbox[rgb]{1.00,0.00,0.00}{1,1,1}{\strut ##1}}}
\expandafter\def\csname PY@tok@mb\endcsname{\def\PY@tc##1{\textcolor[rgb]{0.40,0.40,0.40}{##1}}}
\expandafter\def\csname PY@tok@ss\endcsname{\def\PY@tc##1{\textcolor[rgb]{0.10,0.09,0.49}{##1}}}
\expandafter\def\csname PY@tok@sr\endcsname{\def\PY@tc##1{\textcolor[rgb]{0.73,0.40,0.53}{##1}}}
\expandafter\def\csname PY@tok@mo\endcsname{\def\PY@tc##1{\textcolor[rgb]{0.40,0.40,0.40}{##1}}}
\expandafter\def\csname PY@tok@kd\endcsname{\let\PY@bf=\textbf\def\PY@tc##1{\textcolor[rgb]{0.00,0.50,0.00}{##1}}}
\expandafter\def\csname PY@tok@mi\endcsname{\def\PY@tc##1{\textcolor[rgb]{0.40,0.40,0.40}{##1}}}
\expandafter\def\csname PY@tok@kn\endcsname{\let\PY@bf=\textbf\def\PY@tc##1{\textcolor[rgb]{0.00,0.50,0.00}{##1}}}
\expandafter\def\csname PY@tok@cpf\endcsname{\let\PY@it=\textit\def\PY@tc##1{\textcolor[rgb]{0.25,0.50,0.50}{##1}}}
\expandafter\def\csname PY@tok@kr\endcsname{\let\PY@bf=\textbf\def\PY@tc##1{\textcolor[rgb]{0.00,0.50,0.00}{##1}}}
\expandafter\def\csname PY@tok@s\endcsname{\def\PY@tc##1{\textcolor[rgb]{0.73,0.13,0.13}{##1}}}
\expandafter\def\csname PY@tok@kp\endcsname{\def\PY@tc##1{\textcolor[rgb]{0.00,0.50,0.00}{##1}}}
\expandafter\def\csname PY@tok@w\endcsname{\def\PY@tc##1{\textcolor[rgb]{0.73,0.73,0.73}{##1}}}
\expandafter\def\csname PY@tok@kt\endcsname{\def\PY@tc##1{\textcolor[rgb]{0.69,0.00,0.25}{##1}}}
\expandafter\def\csname PY@tok@sc\endcsname{\def\PY@tc##1{\textcolor[rgb]{0.73,0.13,0.13}{##1}}}
\expandafter\def\csname PY@tok@sb\endcsname{\def\PY@tc##1{\textcolor[rgb]{0.73,0.13,0.13}{##1}}}
\expandafter\def\csname PY@tok@sa\endcsname{\def\PY@tc##1{\textcolor[rgb]{0.73,0.13,0.13}{##1}}}
\expandafter\def\csname PY@tok@k\endcsname{\let\PY@bf=\textbf\def\PY@tc##1{\textcolor[rgb]{0.00,0.50,0.00}{##1}}}
\expandafter\def\csname PY@tok@se\endcsname{\let\PY@bf=\textbf\def\PY@tc##1{\textcolor[rgb]{0.73,0.40,0.13}{##1}}}
\expandafter\def\csname PY@tok@sd\endcsname{\let\PY@it=\textit\def\PY@tc##1{\textcolor[rgb]{0.73,0.13,0.13}{##1}}}

\def\PYZbs{\char`\\}
\def\PYZus{\char`\_}
\def\PYZob{\char`\{}
\def\PYZcb{\char`\}}
\def\PYZca{\char`\^}
\def\PYZam{\char`\&}
\def\PYZlt{\char`\<}
\def\PYZgt{\char`\>}
\def\PYZsh{\char`\#}
\def\PYZpc{\char`\%}
\def\PYZdl{\char`\$}
\def\PYZhy{\char`\-}
\def\PYZsq{\char`\'}
\def\PYZdq{\char`\"}
\def\PYZti{\char`\~}
% for compatibility with earlier versions
\def\PYZat{@}
\def\PYZlb{[}
\def\PYZrb{]}
\makeatother


    % Exact colors from NB
    \definecolor{incolor}{rgb}{0.0, 0.0, 0.5}
    \definecolor{outcolor}{rgb}{0.545, 0.0, 0.0}



    
    % Prevent overflowing lines due to hard-to-break entities
    \sloppy 
    % Setup hyperref package
    \hypersetup{
      breaklinks=true,  % so long urls are correctly broken across lines
      colorlinks=true,
      urlcolor=urlcolor,
      linkcolor=linkcolor,
      citecolor=citecolor,
      }
    % Slightly bigger margins than the latex defaults
    
    \geometry{verbose,tmargin=1in,bmargin=1in,lmargin=1in,rmargin=1in}
    
    

    \begin{document}
    
    
    \maketitle
    
    

    
    \section{CSE 252A Computer Vision I Fall 2018 - Assignment
1}\label{cse-252a-computer-vision-i-fall-2018---assignment-1}

\subsubsection{Instructor: David
Kriegman}\label{instructor-david-kriegman}

\subsubsection{Assignment Published On: Tuesday, October 9,
2018}\label{assignment-published-on-tuesday-october-9-2018}

\subsubsection{Due On: Tuesday, October 23, 2018 11:59
pm}\label{due-on-tuesday-october-23-2018-1159-pm}

\subsection{Instructions}\label{instructions}

\begin{itemize}
\item
  Review the academic integrity and collaboration policies on the course
  website.
\item
  This assignment must be completed individually.
\item
  This assignment contains theoretical and programming exercises. If you
  plan to submit hand written answers for theoretical exercises, please
  be sure your writing is readable and merge those in order with the
  final pdf you create out of this notebook. You could fill the answers
  within the notebook iteself by creating a markdown cell.
\item
  Programming aspects of this assignment must be completed using Python
  in this notebook.
\item
  If you want to modify the skeleton code, you can do so. This has been
  provided just to provide you with a framework for the solution.
\item
  You may use python packages for basic linear algebra (you can use
  numpy or scipy for basic operations), but you may not use packages
  that directly solve the problem.
\item
  If you are unsure about using a specific package or function, then ask
  the instructor and teaching assistants for clarification.
\item
  You must submit this notebook exported as a pdf. You must also submit
  this notebook as .ipynb file.
\item
  You must submit both files (.pdf and .ipynb) on Gradescope. You must
  mark each problem on Gradescope in the pdf.
\item ~
  \subsection{\texorpdfstring{\textbf{Late policy} - 10\% per day late
  penalty after due date up to 3
  days.}{Late policy - 10\% per day late penalty after due date up to 3 days.}}\label{late-policy---10-per-day-late-penalty-after-due-date-up-to-3-days.}
\end{itemize}

    \subsection{Problem 1: Perspective Projection {[}5
pts{]}}\label{problem-1-perspective-projection-5-pts}

Consider a perspective projection where a point \[
P = [\text{x y z}]^T
\] is projected onto an image plane \(\Pi'\) represented by \(k = f'>0\)
as shown in the following figure. The first second and third coordinate
axes are denoted by \(i\), \(j\), \(k\) respectively.

Consider the projection of two rays in the world coordinate system \[
Q1 = [\text{7 -3 1}] + t[\text{8 2 4}]
\] \[
Q2 = [\text{2 -5 9}] + t[\text{8 2 4}]
\] where \(-\infty \leq t \leq -1\).

Calculate the coordinates of the endpoints of the projection of the rays
onto the image plane. Identify the vanishing point based on the
coordinates.

    \subsection{Problem 2: Thin Lens Equation {[}5
pts{]}}\label{problem-2-thin-lens-equation-5-pts}

An illuminated arrow forms a real inverted image of itself at a distance
of \(w = 60\text{ cm}\), measured along the optical axis of a convex
thin lens as shown above. The image is half the size of the object 1.
How far from the object must the lens be placed? Whats is the focal
length of the lens? 2. At what distance from the center of the lens
should the arrow be placed so that the height of the image is the same?
3. What would be the type and location of image formed if the arrow is
placed at a distance of 5 cm along the optical axis from the optical
center?

    \subsection{Problem 3: Affine Projection {[}3
pts{]}}\label{problem-3-affine-projection-3-pts}

Show that the image of a pair of parallel lines in 3D space is a pair of
parallel lines in an affine camera.

    \subsection{Problem 4: Image Formation and Rigid Body Transformations
{[}10
points{]}}\label{problem-4-image-formation-and-rigid-body-transformations-10-points}

In this problem we will practice rigid body transformations and image
formations through the projective and affine camera model. The goal will
be to photograph the following four points
\[^AP_1 = [\text{-1 -0.5 2}]^T\], \[^AP_2 = [\text{1 -0.5 2}]^T\],
\[^AP_3 = [\text{1 0.5 2}]^T\], \[^AP_4 = [\text{-1 0.5 2}]^T\]

To do this we will need two matrices. Recall, first, the following
formula for rigid body transformation \[
^BP = \text{ } ^B_AR\text{ }^AP + \text{ } ^BO_A
\] Where \(^BP\) is the point coordinate in the target (\(B\))
coordinate system. \(^AP\) is the point coordinate in the source (\(A\))
coordinate system. \(^B_AR\) is the rotation matrix from \(A\) to \(B\),
and \(^BO_A\) is the origin of the coordinate system \(A\) expressed in
\(B\) coordinates.

The rotation and translation can be combined into a single 4 \(\times\)
4 extrinsic parameter matrix, \(P_e\), so that
\(^BP = P_e \cdot \text{ }^AP\).

Once transformed, the points can be photographed using the intrinsic
camera matrix, \(P_i\) which is a 3 \(\times\) 4 matrix.

Once these are found, the image of a point, \(^AP\), can be calculated
as \(P_i \cdot P_e \cdot \text{ }^AP\).

We will consider four different settings of focal length, viewing angles
and camera positions below. For each of these calculate:

\begin{enumerate}
\def\labelenumi{\alph{enumi})}
\item
  Extrinsic transformation matrix,
\item
  Intrinsic camera matrix under the perspective camera assumption.
\item
  Intrinsic camera matrix under the affine camera assumption. In
  particular, around what point do you do the taylor series expansion?
\item
  Calculate the image of the four vertices and plot using the supplied
  functions
\end{enumerate}

Your output should look something like the following image (Your output
values might not match, this is just an example) 1. {[}No rigid body
transformation{]}. Focal length = 1. The optical axis of the camera is
aligned with the z-axis. 2. {[}Translation{]}.
\(^BO_A = [\text{0 0 1}]^T\). Focal length = 1. The optical axis of the
camera is aligned with the z-axis. 3. {[}Translation and Rotation{]}.
Focal length = 1. \(^B_AR\) encodes a 30 degrees around the z-axis and
then 60 degrees around the y-axis. \(^BO_A = [\text{0 0 1}]^T\). 4.
{[}Translation and Rotation, long distance{]}. Focal length = 5.
\(^B_AR\) encodes a 30 degrees around the z-axis and then 60 degrees
around the y-axis. \(^BO_A = [\text{0 0 13}]^T\).

\begin{quote}
You can refer the Richard Szeliski starting page 36 for image formation
and the extrinsic matrix.
\end{quote}

\begin{quote}
Intrinsic matrix calculation for perspective and affine camera models
was covered in class and can be referred in slide 3
http://cseweb.ucsd.edu/classes/fa18/cse252A-a/lec3.pdf
\end{quote}

We will not use a full intrinsic camera matrix (e.g. that maps
centimeters to pixels, and defines the coordinates of the center of the
image), but only parameterize this with \(f\), the focal length. In
other words: the only parameter in the intrinsic camera matrix under the
perspective assumption is \(f\), and the only ones under the affine
assumption are: \(f,x_0,y_0,z_0\), where \(x_0,y_0,z_0\) is the center
of the taylor series expansion.

Note that the axis are the same for each row, to facilitate comparison
between the two camera models. Also include: 1. The actual points around
which you did the taylor series expansion for the affine camera models.
2. How did you arrive at these points? 3. How do the projective and
affine camera models differ? Why is this difference smaller for the last
image compared to the second last?

    \begin{Verbatim}[commandchars=\\\{\}]
{\color{incolor}In [{\color{incolor}42}]:} \PY{k+kn}{import} \PY{n+nn}{numpy} \PY{k+kn}{as} \PY{n+nn}{np}
         \PY{k+kn}{import} \PY{n+nn}{matplotlib.pyplot} \PY{k+kn}{as} \PY{n+nn}{plt}
         \PY{k+kn}{import} \PY{n+nn}{math}
         
         
         \PY{c+c1}{\PYZsh{} convert points from euclidian to homogeneous}
         \PY{k}{def} \PY{n+nf}{to\PYZus{}homog}\PY{p}{(}\PY{n}{points}\PY{p}{)}\PY{p}{:}
             \PY{k}{if} \PY{n+nb}{len}\PY{p}{(}\PY{n}{points}\PY{p}{)} \PY{o}{==} \PY{l+m+mi}{0} \PY{o+ow}{or} \PY{n}{points} \PY{o+ow}{is} \PY{n+nb+bp}{None}\PY{p}{:}
                 \PY{k}{return} \PY{n+nb+bp}{None}
             
             \PY{k}{if} \PY{n+nb}{len}\PY{p}{(}\PY{n}{points}\PY{o}{.}\PY{n}{shape}\PY{p}{)} \PY{o}{==} \PY{l+m+mi}{1}\PY{p}{:}
                 \PY{n}{oneLine} \PY{o}{=} \PY{n}{np}\PY{o}{.}\PY{n}{array}\PY{p}{(}\PY{p}{[}\PY{l+m+mi}{1}\PY{p}{]}\PY{p}{)}\PY{o}{.}\PY{n}{T} \PY{c+c1}{\PYZsh{} add \PYZsq{}1\PYZsq{}}
                 \PY{k}{return} \PY{n}{np}\PY{o}{.}\PY{n}{append}\PY{p}{(}\PY{n}{points}\PY{p}{,} \PY{n}{oneLine}\PY{p}{)}
             \PY{k}{else}\PY{p}{:}
                 \PY{n}{oneLine} \PY{o}{=} \PY{n}{np}\PY{o}{.}\PY{n}{ones}\PY{p}{(}\PY{n}{points}\PY{o}{.}\PY{n}{shape}\PY{p}{[}\PY{l+m+mi}{1}\PY{p}{]}\PY{p}{)}
                 \PY{k}{return} \PY{n}{np}\PY{o}{.}\PY{n}{vstack}\PY{p}{(}\PY{p}{(}\PY{n}{points}\PY{p}{,} \PY{n}{oneLine}\PY{p}{)}\PY{p}{)} \PY{c+c1}{\PYZsh{} add a line of \PYZsq{}1\PYZsq{}s if using matrices}
         
         
         \PY{c+c1}{\PYZsh{} convert points from homogeneous to euclidian}
         \PY{k}{def} \PY{n+nf}{from\PYZus{}homog}\PY{p}{(}\PY{n}{points\PYZus{}homog}\PY{p}{)}\PY{p}{:} 
             \PY{k}{if} \PY{n+nb}{len}\PY{p}{(}\PY{n}{points\PYZus{}homog}\PY{p}{)} \PY{o}{==} \PY{l+m+mi}{0} \PY{o+ow}{or} \PY{n}{points\PYZus{}homog} \PY{o+ow}{is} \PY{n+nb+bp}{None}\PY{p}{:}
                 \PY{k}{return} \PY{n+nb+bp}{None}
             
             \PY{k}{return} \PY{n}{points\PYZus{}homog}\PY{p}{[}\PY{p}{:}\PY{o}{\PYZhy{}}\PY{l+m+mi}{1}\PY{p}{]}\PY{o}{/}\PY{n}{points\PYZus{}homog}\PY{p}{[}\PY{o}{\PYZhy{}}\PY{l+m+mi}{1}\PY{p}{]} \PY{c+c1}{\PYZsh{} [x/z, y/z]}
         
         \PY{c+c1}{\PYZsh{} project 3D euclidian points to 2D euclidian}
         \PY{k}{def} \PY{n+nf}{project\PYZus{}points}\PY{p}{(}\PY{n}{P\PYZus{}int}\PY{p}{,} \PY{n}{P\PYZus{}ext}\PY{p}{,} \PY{n}{pts}\PY{p}{)}\PY{p}{:}
             \PY{c+c1}{\PYZsh{} write your code here}
             \PY{k}{return} \PY{n}{from\PYZus{}homog}\PY{p}{(}\PY{n}{np}\PY{o}{.}\PY{n}{matmul}\PY{p}{(}\PY{n}{P\PYZus{}int}\PY{p}{,} \PY{n}{np}\PY{o}{.}\PY{n}{matmul}\PY{p}{(}\PY{n}{P\PYZus{}ext}\PY{p}{,} \PY{n}{to\PYZus{}homog}\PY{p}{(}\PY{n}{pts}\PY{p}{)}\PY{p}{)}\PY{p}{)}\PY{p}{)} \PY{c+c1}{\PYZsh{} results : Pi * Pe * points}
             
         
         
         \PY{c+c1}{\PYZsh{} Change the three matrices for the four cases as described in the problem}
         \PY{c+c1}{\PYZsh{} in the four camera functions geiven below. Make sure that we can see the formula}
         \PY{c+c1}{\PYZsh{} (if one exists) being used to fill in the matrices. Feel free to document with}
         \PY{c+c1}{\PYZsh{} comments any thing you feel the need to explain. }
         
         \PY{k}{def} \PY{n+nf}{camera1}\PY{p}{(}\PY{p}{)}\PY{p}{:}
             \PY{c+c1}{\PYZsh{} write your code here}
             \PY{n}{f} \PY{o}{=} \PY{l+m+mf}{1.0} \PY{c+c1}{\PYZsh{} focal length}
             \PY{n}{P} \PY{o}{=} \PY{n}{np}\PY{o}{.}\PY{n}{array}\PY{p}{(}\PY{p}{[}\PY{l+m+mi}{0}\PY{p}{,} \PY{l+m+mi}{0}\PY{p}{,} \PY{l+m+mi}{2}\PY{p}{,} \PY{l+m+mi}{1}\PY{p}{]}\PY{p}{)} \PY{c+c1}{\PYZsh{} original points}
             \PY{n}{P\PYZus{}int\PYZus{}proj} \PY{o}{=} \PY{n}{np}\PY{o}{.}\PY{n}{eye}\PY{p}{(}\PY{l+m+mi}{3}\PY{p}{,}\PY{l+m+mi}{4}\PY{p}{)}
             \PY{n}{P\PYZus{}int\PYZus{}affine} \PY{o}{=} \PY{n}{np}\PY{o}{.}\PY{n}{eye}\PY{p}{(}\PY{l+m+mi}{3}\PY{p}{,}\PY{l+m+mi}{4}\PY{p}{)}
             \PY{n}{P\PYZus{}ext} \PY{o}{=} \PY{n}{np}\PY{o}{.}\PY{n}{eye}\PY{p}{(}\PY{l+m+mi}{4}\PY{p}{,}\PY{l+m+mi}{4}\PY{p}{)}
             
             \PY{n}{P\PYZus{}ext} \PY{o}{=} \PY{n}{np}\PY{o}{.}\PY{n}{vstack}\PY{p}{(}\PY{p}{(}\PY{n}{np}\PY{o}{.}\PY{n}{hstack}\PY{p}{(}\PY{p}{(}\PY{n}{np}\PY{o}{.}\PY{n}{eye}\PY{p}{(}\PY{l+m+mi}{3}\PY{p}{,}\PY{l+m+mi}{3}\PY{p}{)}\PY{p}{,} \PY{n}{np}\PY{o}{.}\PY{n}{zeros}\PY{p}{(}\PY{p}{(}\PY{l+m+mi}{3}\PY{p}{,} \PY{l+m+mi}{1}\PY{p}{)}\PY{p}{)}\PY{p}{)}\PY{p}{)}\PY{p}{,} \PY{n}{np}\PY{o}{.}\PY{n}{array}\PY{p}{(}\PY{p}{[}\PY{l+m+mi}{0}\PY{p}{,} \PY{l+m+mi}{0}\PY{p}{,} \PY{l+m+mi}{0} \PY{p}{,}\PY{l+m+mi}{1}\PY{p}{]}\PY{p}{)}\PY{p}{)}\PY{p}{)} \PY{c+c1}{\PYZsh{} modify P\PYZus{}ext matrix by adding}
                                                                                                     \PY{c+c1}{\PYZsh{} translation information}
             
             \PY{n}{P\PYZus{}} \PY{o}{=} \PY{n}{np}\PY{o}{.}\PY{n}{matmul}\PY{p}{(}\PY{n}{P\PYZus{}ext}\PY{p}{,} \PY{n}{P}\PY{p}{)} \PY{c+c1}{\PYZsh{} perform P\PYZus{}ext matrix to remove camera artifacts}
             \PY{n}{x0} \PY{o}{=} \PY{n}{P\PYZus{}}\PY{p}{[}\PY{l+m+mi}{0}\PY{p}{]}
             \PY{n}{y0} \PY{o}{=} \PY{n}{P\PYZus{}}\PY{p}{[}\PY{l+m+mi}{1}\PY{p}{]}
             \PY{n}{z0} \PY{o}{=} \PY{n}{P\PYZus{}}\PY{p}{[}\PY{l+m+mi}{2}\PY{p}{]}
             \PY{c+c1}{\PYZsh{}print(\PYZdq{}Point: \PYZdq{}, x0, y0, z0) \PYZsh{} actual point for affine camera model}
             
             \PY{c+c1}{\PYZsh{} generate the Pi\PYZus{}proj matrix}
             \PY{n}{P\PYZus{}int\PYZus{}proj}\PY{p}{[}\PY{l+m+mi}{0}\PY{p}{,} \PY{l+m+mi}{0}\PY{p}{]}\PY{p}{,} \PY{n}{P\PYZus{}int\PYZus{}proj}\PY{p}{[}\PY{l+m+mi}{1}\PY{p}{,} \PY{l+m+mi}{1}\PY{p}{]} \PY{o}{=} \PY{n}{f}\PY{p}{,} \PY{n}{f} 
             \PY{c+c1}{\PYZsh{} generate the Pi\PYZus{}affine matrix}
             \PY{n}{P\PYZus{}int\PYZus{}affine}\PY{p}{[}\PY{l+m+mi}{0}\PY{p}{,} \PY{l+m+mi}{0}\PY{p}{]}\PY{p}{,} \PY{n}{P\PYZus{}int\PYZus{}affine}\PY{p}{[}\PY{l+m+mi}{1}\PY{p}{,} \PY{l+m+mi}{1}\PY{p}{]}\PY{p}{,} \PY{n}{P\PYZus{}int\PYZus{}affine}\PY{p}{[}\PY{l+m+mi}{2}\PY{p}{,} \PY{l+m+mi}{2}\PY{p}{]}\PY{p}{,} \PY{n}{P\PYZus{}int\PYZus{}affine}\PY{p}{[}\PY{l+m+mi}{2}\PY{p}{,} \PY{l+m+mi}{3}\PY{p}{]} \PY{o}{=} \PY{n}{f}\PY{o}{/}\PY{n}{z0}\PY{p}{,} \PY{n}{f}\PY{o}{/}\PY{n}{z0}\PY{p}{,} \PY{l+m+mi}{0}\PY{p}{,} \PY{l+m+mi}{1} 
             \PY{n}{P\PYZus{}int\PYZus{}affine}\PY{p}{[}\PY{l+m+mi}{0}\PY{p}{:}\PY{l+m+mi}{2}\PY{p}{,} \PY{l+m+mi}{2}\PY{p}{:}\PY{p}{]} \PY{o}{=} \PY{n}{np}\PY{o}{.}\PY{n}{array}\PY{p}{(}\PY{p}{[}\PY{p}{[}\PY{o}{\PYZhy{}}\PY{n}{f}\PY{o}{*}\PY{n}{x0} \PY{o}{/} \PY{p}{(}\PY{n}{z0}\PY{p}{)}\PY{o}{*}\PY{o}{*}\PY{l+m+mi}{2}\PY{p}{,} \PY{n}{f}\PY{o}{*}\PY{n}{x0}\PY{o}{/}\PY{n}{z0}\PY{p}{]}\PY{p}{,} \PY{p}{[}\PY{o}{\PYZhy{}}\PY{n}{f}\PY{o}{*}\PY{n}{y0} \PY{o}{/} \PY{p}{(}\PY{n}{z0}\PY{p}{)}\PY{o}{*}\PY{o}{*}\PY{l+m+mi}{2}\PY{p}{,} \PY{n}{f}\PY{o}{*}\PY{n}{y0}\PY{o}{/}\PY{n}{z0}\PY{p}{]}\PY{p}{]}\PY{p}{)}
             
             
             \PY{k}{return} \PY{n}{P\PYZus{}int\PYZus{}proj}\PY{p}{,} \PY{n}{P\PYZus{}int\PYZus{}affine}\PY{p}{,} \PY{n}{P\PYZus{}ext}
         
         \PY{k}{def} \PY{n+nf}{camera2}\PY{p}{(}\PY{p}{)}\PY{p}{:}
             \PY{c+c1}{\PYZsh{} write your code here}
             \PY{n}{f} \PY{o}{=} \PY{l+m+mf}{1.0} \PY{c+c1}{\PYZsh{} focal length}
             \PY{n}{P} \PY{o}{=} \PY{n}{np}\PY{o}{.}\PY{n}{array}\PY{p}{(}\PY{p}{[}\PY{l+m+mi}{0}\PY{p}{,} \PY{l+m+mi}{0}\PY{p}{,} \PY{l+m+mi}{2}\PY{p}{,} \PY{l+m+mi}{1}\PY{p}{]}\PY{p}{)} \PY{c+c1}{\PYZsh{} original points}
             
             \PY{n}{P\PYZus{}int\PYZus{}proj} \PY{o}{=} \PY{n}{np}\PY{o}{.}\PY{n}{eye}\PY{p}{(}\PY{l+m+mi}{3}\PY{p}{,}\PY{l+m+mi}{4}\PY{p}{)}
             \PY{n}{P\PYZus{}int\PYZus{}affine} \PY{o}{=} \PY{n}{np}\PY{o}{.}\PY{n}{eye}\PY{p}{(}\PY{l+m+mi}{3}\PY{p}{,}\PY{l+m+mi}{4}\PY{p}{)}
             \PY{n}{P\PYZus{}ext} \PY{o}{=} \PY{n}{np}\PY{o}{.}\PY{n}{eye}\PY{p}{(}\PY{l+m+mi}{4}\PY{p}{,}\PY{l+m+mi}{4}\PY{p}{)}
             
             \PY{c+c1}{\PYZsh{} generate P\PYZus{}ext matrix}
             \PY{n}{P\PYZus{}ext}\PY{p}{[}\PY{l+m+mi}{2}\PY{p}{,} \PY{l+m+mi}{3}\PY{p}{]} \PY{o}{=} \PY{l+m+mf}{1.0}
             
             \PY{n}{P\PYZus{}} \PY{o}{=} \PY{n}{np}\PY{o}{.}\PY{n}{matmul}\PY{p}{(}\PY{n}{P\PYZus{}ext}\PY{p}{,} \PY{n}{P}\PY{p}{)} \PY{c+c1}{\PYZsh{} perform P\PYZus{}ext matrix on points}
             \PY{n}{x0} \PY{o}{=} \PY{n}{P\PYZus{}}\PY{p}{[}\PY{l+m+mi}{0}\PY{p}{]}
             \PY{n}{y0} \PY{o}{=} \PY{n}{P\PYZus{}}\PY{p}{[}\PY{l+m+mi}{1}\PY{p}{]}
             \PY{n}{z0} \PY{o}{=} \PY{n}{P\PYZus{}}\PY{p}{[}\PY{l+m+mi}{2}\PY{p}{]}
             \PY{c+c1}{\PYZsh{}print(\PYZdq{}Point: \PYZdq{}, x0, y0, z0) \PYZsh{} actual point for affine camera model}
             \PY{c+c1}{\PYZsh{} generate the Pi\PYZus{}proj matrix}
             \PY{n}{P\PYZus{}int\PYZus{}proj}\PY{p}{[}\PY{l+m+mi}{0}\PY{p}{,} \PY{l+m+mi}{0}\PY{p}{]}\PY{p}{,} \PY{n}{P\PYZus{}int\PYZus{}proj}\PY{p}{[}\PY{l+m+mi}{1}\PY{p}{,} \PY{l+m+mi}{1}\PY{p}{]} \PY{o}{=} \PY{n}{f}\PY{p}{,} \PY{n}{f}
             \PY{c+c1}{\PYZsh{} generate the Pi\PYZus{}affine matrix}
             \PY{n}{P\PYZus{}int\PYZus{}affine}\PY{p}{[}\PY{l+m+mi}{0}\PY{p}{,} \PY{l+m+mi}{0}\PY{p}{]}\PY{p}{,} \PY{n}{P\PYZus{}int\PYZus{}affine}\PY{p}{[}\PY{l+m+mi}{1}\PY{p}{,} \PY{l+m+mi}{1}\PY{p}{]}\PY{p}{,} \PY{n}{P\PYZus{}int\PYZus{}affine}\PY{p}{[}\PY{l+m+mi}{2}\PY{p}{,} \PY{l+m+mi}{2}\PY{p}{]}\PY{p}{,} \PY{n}{P\PYZus{}int\PYZus{}affine}\PY{p}{[}\PY{l+m+mi}{2}\PY{p}{,} \PY{l+m+mi}{3}\PY{p}{]} \PY{o}{=} \PY{n}{f}\PY{o}{/}\PY{n}{z0}\PY{p}{,} \PY{n}{f}\PY{o}{/}\PY{n}{z0}\PY{p}{,} \PY{l+m+mi}{0}\PY{p}{,} \PY{l+m+mi}{1} 
             \PY{n}{P\PYZus{}int\PYZus{}affine}\PY{p}{[}\PY{l+m+mi}{0}\PY{p}{:}\PY{l+m+mi}{2}\PY{p}{,} \PY{l+m+mi}{2}\PY{p}{:}\PY{p}{]} \PY{o}{=} \PY{n}{np}\PY{o}{.}\PY{n}{array}\PY{p}{(}\PY{p}{[}\PY{p}{[}\PY{o}{\PYZhy{}}\PY{n}{f}\PY{o}{*}\PY{n}{x0} \PY{o}{/} \PY{p}{(}\PY{n}{z0}\PY{p}{)}\PY{o}{*}\PY{o}{*}\PY{l+m+mi}{2}\PY{p}{,} \PY{n}{f}\PY{o}{*}\PY{n}{x0}\PY{o}{/}\PY{n}{z0}\PY{p}{]}\PY{p}{,} \PY{p}{[}\PY{o}{\PYZhy{}}\PY{n}{f}\PY{o}{*}\PY{n}{y0} \PY{o}{/} \PY{p}{(}\PY{n}{z0}\PY{p}{)}\PY{o}{*}\PY{o}{*}\PY{l+m+mi}{2}\PY{p}{,} \PY{n}{f}\PY{o}{*}\PY{n}{y0}\PY{o}{/}\PY{n}{z0}\PY{p}{]}\PY{p}{]}\PY{p}{)}
             
             \PY{k}{return} \PY{n}{P\PYZus{}int\PYZus{}proj}\PY{p}{,} \PY{n}{P\PYZus{}int\PYZus{}affine}\PY{p}{,} \PY{n}{P\PYZus{}ext}
         
         \PY{k}{def} \PY{n+nf}{camera3}\PY{p}{(}\PY{p}{)}\PY{p}{:}
             \PY{c+c1}{\PYZsh{} write your code here}
             \PY{n}{f} \PY{o}{=} \PY{l+m+mf}{1.0} \PY{c+c1}{\PYZsh{} focal length}
             \PY{n}{P} \PY{o}{=} \PY{n}{np}\PY{o}{.}\PY{n}{array}\PY{p}{(}\PY{p}{[}\PY{l+m+mi}{0}\PY{p}{,} \PY{l+m+mi}{0}\PY{p}{,} \PY{l+m+mi}{2}\PY{p}{,} \PY{l+m+mi}{1}\PY{p}{]}\PY{p}{)} \PY{c+c1}{\PYZsh{} original point in world coordinate system}
             
             \PY{n}{P\PYZus{}int\PYZus{}proj} \PY{o}{=} \PY{n}{np}\PY{o}{.}\PY{n}{eye}\PY{p}{(}\PY{l+m+mi}{3}\PY{p}{,}\PY{l+m+mi}{4}\PY{p}{)}
             \PY{n}{P\PYZus{}int\PYZus{}affine} \PY{o}{=} \PY{n}{np}\PY{o}{.}\PY{n}{eye}\PY{p}{(}\PY{l+m+mi}{3}\PY{p}{,}\PY{l+m+mi}{4}\PY{p}{)}
             \PY{n}{P\PYZus{}ext} \PY{o}{=} \PY{n}{np}\PY{o}{.}\PY{n}{eye}\PY{p}{(}\PY{l+m+mi}{4}\PY{p}{,}\PY{l+m+mi}{4}\PY{p}{)}
             
             \PY{c+c1}{\PYZsh{} rotation around z\PYZhy{}axis}
             \PY{n}{Rz} \PY{o}{=} \PY{n}{np}\PY{o}{.}\PY{n}{array}\PY{p}{(}\PY{p}{[}\PY{p}{[}\PY{n}{np}\PY{o}{.}\PY{n}{cos}\PY{p}{(}\PY{n}{np}\PY{o}{.}\PY{n}{deg2rad}\PY{p}{(}\PY{l+m+mi}{30}\PY{p}{)}\PY{p}{)}\PY{p}{,} \PY{o}{\PYZhy{}}\PY{n}{np}\PY{o}{.}\PY{n}{sin}\PY{p}{(}\PY{n}{np}\PY{o}{.}\PY{n}{deg2rad}\PY{p}{(}\PY{l+m+mi}{30}\PY{p}{)}\PY{p}{)}\PY{p}{,} \PY{l+m+mi}{0}\PY{p}{]}\PY{p}{,} \PY{p}{[}\PY{n}{np}\PY{o}{.}\PY{n}{sin}\PY{p}{(}\PY{n}{np}\PY{o}{.}\PY{n}{deg2rad}\PY{p}{(}\PY{l+m+mi}{30}\PY{p}{)}\PY{p}{)}\PY{p}{,} \PY{n}{np}\PY{o}{.}\PY{n}{cos}\PY{p}{(}\PY{n}{np}\PY{o}{.}\PY{n}{deg2rad}\PY{p}{(}\PY{l+m+mi}{30}\PY{p}{)}\PY{p}{)}\PY{p}{,} \PY{l+m+mi}{0}\PY{p}{]}\PY{p}{,} \PY{p}{[}\PY{l+m+mi}{0}\PY{p}{,} \PY{l+m+mi}{0}\PY{p}{,} \PY{l+m+mi}{1}\PY{p}{]}\PY{p}{]}\PY{p}{)}
             \PY{c+c1}{\PYZsh{} rotation around y\PYZhy{}axi}
             \PY{n}{Ry} \PY{o}{=} \PY{n}{np}\PY{o}{.}\PY{n}{array}\PY{p}{(}\PY{p}{[}\PY{p}{[}\PY{n}{np}\PY{o}{.}\PY{n}{cos}\PY{p}{(}\PY{n}{np}\PY{o}{.}\PY{n}{deg2rad}\PY{p}{(}\PY{l+m+mi}{60}\PY{p}{)}\PY{p}{)}\PY{p}{,} \PY{l+m+mi}{0}\PY{p}{,} \PY{n}{np}\PY{o}{.}\PY{n}{sin}\PY{p}{(}\PY{n}{np}\PY{o}{.}\PY{n}{deg2rad}\PY{p}{(}\PY{l+m+mi}{60}\PY{p}{)}\PY{p}{)}\PY{p}{]}\PY{p}{,} \PY{p}{[}\PY{l+m+mi}{0}\PY{p}{,} \PY{l+m+mi}{1}\PY{p}{,} \PY{l+m+mi}{0}\PY{p}{]}\PY{p}{,} \PY{p}{[}\PY{o}{\PYZhy{}}\PY{n}{np}\PY{o}{.}\PY{n}{sin}\PY{p}{(}\PY{n}{np}\PY{o}{.}\PY{n}{deg2rad}\PY{p}{(}\PY{l+m+mi}{60}\PY{p}{)}\PY{p}{)}\PY{p}{,} \PY{l+m+mi}{0}\PY{p}{,} \PY{n}{np}\PY{o}{.}\PY{n}{cos}\PY{p}{(}\PY{n}{np}\PY{o}{.}\PY{n}{deg2rad}\PY{p}{(}\PY{l+m+mi}{60}\PY{p}{)}\PY{p}{)}\PY{p}{]}\PY{p}{]}\PY{p}{)}
             \PY{c+c1}{\PYZsh{} generate rotation matrix}
             \PY{n}{R} \PY{o}{=} \PY{n}{np}\PY{o}{.}\PY{n}{matmul}\PY{p}{(}\PY{n}{Ry}\PY{p}{,} \PY{n}{Rz}\PY{p}{)}
             \PY{c+c1}{\PYZsh{}generate P\PYZus{}ext matrix}
             \PY{n}{P\PYZus{}ext}\PY{p}{[}\PY{l+m+mi}{0}\PY{p}{:}\PY{l+m+mi}{3}\PY{p}{,} \PY{l+m+mi}{0}\PY{p}{:}\PY{l+m+mi}{3}\PY{p}{]} \PY{o}{=} \PY{n}{R}
             \PY{n}{P\PYZus{}ext}\PY{p}{[}\PY{l+m+mi}{2}\PY{p}{,} \PY{l+m+mi}{3}\PY{p}{]} \PY{o}{=} \PY{l+m+mf}{1.0}
             
             \PY{n}{P\PYZus{}} \PY{o}{=} \PY{n}{np}\PY{o}{.}\PY{n}{matmul}\PY{p}{(}\PY{n}{P\PYZus{}ext}\PY{p}{,} \PY{n}{P}\PY{p}{)}
             \PY{n}{x0} \PY{o}{=} \PY{n}{P\PYZus{}}\PY{p}{[}\PY{l+m+mi}{0}\PY{p}{]}
             \PY{n}{y0} \PY{o}{=} \PY{n}{P\PYZus{}}\PY{p}{[}\PY{l+m+mi}{1}\PY{p}{]}
             \PY{n}{z0} \PY{o}{=} \PY{n}{P\PYZus{}}\PY{p}{[}\PY{l+m+mi}{2}\PY{p}{]}
             \PY{c+c1}{\PYZsh{}print(\PYZdq{}Point: \PYZdq{}, x0, y0, z0) \PYZsh{} actual point for affine camera model}
             \PY{c+c1}{\PYZsh{} generate P\PYZus{}int\PYZus{}proj matrix}
             \PY{n}{P\PYZus{}int\PYZus{}proj}\PY{p}{[}\PY{l+m+mi}{0}\PY{p}{,} \PY{l+m+mi}{0}\PY{p}{]}\PY{p}{,} \PY{n}{P\PYZus{}int\PYZus{}proj}\PY{p}{[}\PY{l+m+mi}{1}\PY{p}{,} \PY{l+m+mi}{1}\PY{p}{]} \PY{o}{=} \PY{n}{f}\PY{p}{,} \PY{n}{f}
             \PY{c+c1}{\PYZsh{} generate P\PYZus{}int\PYZus{}affine matrix}
             \PY{n}{P\PYZus{}int\PYZus{}affine}\PY{p}{[}\PY{l+m+mi}{0}\PY{p}{,} \PY{l+m+mi}{0}\PY{p}{]}\PY{p}{,} \PY{n}{P\PYZus{}int\PYZus{}affine}\PY{p}{[}\PY{l+m+mi}{1}\PY{p}{,} \PY{l+m+mi}{1}\PY{p}{]}\PY{p}{,} \PY{n}{P\PYZus{}int\PYZus{}affine}\PY{p}{[}\PY{l+m+mi}{2}\PY{p}{,} \PY{l+m+mi}{2}\PY{p}{]}\PY{p}{,} \PY{n}{P\PYZus{}int\PYZus{}affine}\PY{p}{[}\PY{l+m+mi}{2}\PY{p}{,} \PY{l+m+mi}{3}\PY{p}{]} \PY{o}{=} \PY{n}{f}\PY{o}{/}\PY{n}{z0}\PY{p}{,} \PY{n}{f}\PY{o}{/}\PY{n}{z0}\PY{p}{,} \PY{l+m+mi}{0}\PY{p}{,} \PY{l+m+mi}{1} 
             \PY{n}{P\PYZus{}int\PYZus{}affine}\PY{p}{[}\PY{l+m+mi}{0}\PY{p}{:}\PY{l+m+mi}{2}\PY{p}{,} \PY{l+m+mi}{2}\PY{p}{:}\PY{p}{]} \PY{o}{=} \PY{n}{np}\PY{o}{.}\PY{n}{array}\PY{p}{(}\PY{p}{[}\PY{p}{[}\PY{o}{\PYZhy{}}\PY{n}{f}\PY{o}{*}\PY{n}{x0} \PY{o}{/} \PY{p}{(}\PY{n}{z0}\PY{p}{)}\PY{o}{*}\PY{o}{*}\PY{l+m+mi}{2}\PY{p}{,} \PY{n}{f}\PY{o}{*}\PY{n}{x0}\PY{o}{/}\PY{n}{z0}\PY{p}{]}\PY{p}{,} \PY{p}{[}\PY{o}{\PYZhy{}}\PY{n}{f}\PY{o}{*}\PY{n}{y0} \PY{o}{/} \PY{p}{(}\PY{n}{z0}\PY{p}{)}\PY{o}{*}\PY{o}{*}\PY{l+m+mi}{2}\PY{p}{,} \PY{n}{f}\PY{o}{*}\PY{n}{y0}\PY{o}{/}\PY{n}{z0}\PY{p}{]}\PY{p}{]}\PY{p}{)}
             
             \PY{k}{return} \PY{n}{P\PYZus{}int\PYZus{}proj}\PY{p}{,} \PY{n}{P\PYZus{}int\PYZus{}affine}\PY{p}{,} \PY{n}{P\PYZus{}ext}
         
         
         \PY{k}{def} \PY{n+nf}{camera4}\PY{p}{(}\PY{p}{)}\PY{p}{:}    
             \PY{c+c1}{\PYZsh{} write your code here}
             \PY{n}{f} \PY{o}{=} \PY{l+m+mf}{5.0} \PY{c+c1}{\PYZsh{} focal length}
             \PY{n}{P} \PY{o}{=} \PY{n}{np}\PY{o}{.}\PY{n}{array}\PY{p}{(}\PY{p}{[}\PY{l+m+mi}{0}\PY{p}{,} \PY{l+m+mi}{0}\PY{p}{,} \PY{l+m+mi}{2}\PY{p}{,} \PY{l+m+mi}{1}\PY{p}{]}\PY{p}{)} \PY{c+c1}{\PYZsh{} original point in world cooridnates system}
             
             \PY{n}{P\PYZus{}int\PYZus{}proj} \PY{o}{=} \PY{n}{np}\PY{o}{.}\PY{n}{eye}\PY{p}{(}\PY{l+m+mi}{3}\PY{p}{,}\PY{l+m+mi}{4}\PY{p}{)}
             \PY{n}{P\PYZus{}int\PYZus{}affine} \PY{o}{=} \PY{n}{np}\PY{o}{.}\PY{n}{eye}\PY{p}{(}\PY{l+m+mi}{3}\PY{p}{,}\PY{l+m+mi}{4}\PY{p}{)}
             \PY{n}{P\PYZus{}ext} \PY{o}{=} \PY{n}{np}\PY{o}{.}\PY{n}{eye}\PY{p}{(}\PY{l+m+mi}{4}\PY{p}{,}\PY{l+m+mi}{4}\PY{p}{)}
             
             \PY{n}{Rz} \PY{o}{=} \PY{n}{np}\PY{o}{.}\PY{n}{array}\PY{p}{(}\PY{p}{[}\PY{p}{[}\PY{n}{np}\PY{o}{.}\PY{n}{cos}\PY{p}{(}\PY{n}{np}\PY{o}{.}\PY{n}{deg2rad}\PY{p}{(}\PY{l+m+mi}{30}\PY{p}{)}\PY{p}{)}\PY{p}{,} \PY{o}{\PYZhy{}}\PY{n}{np}\PY{o}{.}\PY{n}{sin}\PY{p}{(}\PY{n}{np}\PY{o}{.}\PY{n}{deg2rad}\PY{p}{(}\PY{l+m+mi}{30}\PY{p}{)}\PY{p}{)}\PY{p}{,} \PY{l+m+mi}{0}\PY{p}{]}\PY{p}{,} \PY{p}{[}\PY{n}{np}\PY{o}{.}\PY{n}{sin}\PY{p}{(}\PY{n}{np}\PY{o}{.}\PY{n}{deg2rad}\PY{p}{(}\PY{l+m+mi}{30}\PY{p}{)}\PY{p}{)}\PY{p}{,} \PY{n}{np}\PY{o}{.}\PY{n}{cos}\PY{p}{(}\PY{n}{np}\PY{o}{.}\PY{n}{deg2rad}\PY{p}{(}\PY{l+m+mi}{30}\PY{p}{)}\PY{p}{)}\PY{p}{,} \PY{l+m+mi}{0}\PY{p}{]}\PY{p}{,} \PY{p}{[}\PY{l+m+mi}{0}\PY{p}{,} \PY{l+m+mi}{0}\PY{p}{,} \PY{l+m+mi}{1}\PY{p}{]}\PY{p}{]}\PY{p}{)}
             \PY{n}{Ry} \PY{o}{=} \PY{n}{np}\PY{o}{.}\PY{n}{array}\PY{p}{(}\PY{p}{[}\PY{p}{[}\PY{n}{np}\PY{o}{.}\PY{n}{cos}\PY{p}{(}\PY{n}{np}\PY{o}{.}\PY{n}{deg2rad}\PY{p}{(}\PY{l+m+mi}{60}\PY{p}{)}\PY{p}{)}\PY{p}{,} \PY{l+m+mi}{0}\PY{p}{,} \PY{n}{np}\PY{o}{.}\PY{n}{sin}\PY{p}{(}\PY{n}{np}\PY{o}{.}\PY{n}{deg2rad}\PY{p}{(}\PY{l+m+mi}{60}\PY{p}{)}\PY{p}{)}\PY{p}{]}\PY{p}{,} \PY{p}{[}\PY{l+m+mi}{0}\PY{p}{,} \PY{l+m+mi}{1}\PY{p}{,} \PY{l+m+mi}{0}\PY{p}{]}\PY{p}{,} \PY{p}{[}\PY{o}{\PYZhy{}}\PY{n}{np}\PY{o}{.}\PY{n}{sin}\PY{p}{(}\PY{n}{np}\PY{o}{.}\PY{n}{deg2rad}\PY{p}{(}\PY{l+m+mi}{60}\PY{p}{)}\PY{p}{)}\PY{p}{,} \PY{l+m+mi}{0}\PY{p}{,} \PY{n}{np}\PY{o}{.}\PY{n}{cos}\PY{p}{(}\PY{n}{np}\PY{o}{.}\PY{n}{deg2rad}\PY{p}{(}\PY{l+m+mi}{60}\PY{p}{)}\PY{p}{)}\PY{p}{]}\PY{p}{]}\PY{p}{)}
             \PY{c+c1}{\PYZsh{} generate rotation matrix}
             \PY{n}{R} \PY{o}{=} \PY{n}{np}\PY{o}{.}\PY{n}{matmul}\PY{p}{(}\PY{n}{Ry}\PY{p}{,} \PY{n}{Rz}\PY{p}{)}
             \PY{n}{P\PYZus{}ext}\PY{p}{[}\PY{l+m+mi}{0}\PY{p}{:}\PY{l+m+mi}{3}\PY{p}{,} \PY{l+m+mi}{0}\PY{p}{:}\PY{l+m+mi}{3}\PY{p}{]} \PY{o}{=} \PY{n}{R}
             \PY{n}{P\PYZus{}ext}\PY{p}{[}\PY{l+m+mi}{2}\PY{p}{,} \PY{l+m+mi}{3}\PY{p}{]} \PY{o}{=} \PY{l+m+mf}{13.0} \PY{c+c1}{\PYZsh{} translation information}
             
             \PY{n}{P\PYZus{}} \PY{o}{=} \PY{n}{np}\PY{o}{.}\PY{n}{matmul}\PY{p}{(}\PY{n}{P\PYZus{}ext}\PY{p}{,} \PY{n}{P}\PY{p}{)}
             \PY{n}{x0} \PY{o}{=} \PY{n}{P\PYZus{}}\PY{p}{[}\PY{l+m+mi}{0}\PY{p}{]}
             \PY{n}{y0} \PY{o}{=} \PY{n}{P\PYZus{}}\PY{p}{[}\PY{l+m+mi}{1}\PY{p}{]}
             \PY{n}{z0} \PY{o}{=} \PY{n}{P\PYZus{}}\PY{p}{[}\PY{l+m+mi}{2}\PY{p}{]}
             \PY{c+c1}{\PYZsh{}print(\PYZdq{}Point: \PYZdq{}, x0, y0, z0) \PYZsh{} actual point for affine camera model}
             \PY{c+c1}{\PYZsh{} generate p\PYZus{}int perspective projection}
             \PY{n}{P\PYZus{}int\PYZus{}proj}\PY{p}{[}\PY{l+m+mi}{0}\PY{p}{,} \PY{l+m+mi}{0}\PY{p}{]}\PY{p}{,} \PY{n}{P\PYZus{}int\PYZus{}proj}\PY{p}{[}\PY{l+m+mi}{1}\PY{p}{,} \PY{l+m+mi}{1}\PY{p}{]} \PY{o}{=} \PY{n}{f}\PY{p}{,} \PY{n}{f}
             \PY{c+c1}{\PYZsh{} generate p\PYZus{}int in affine camera model}
             \PY{n}{P\PYZus{}int\PYZus{}affine}\PY{p}{[}\PY{l+m+mi}{0}\PY{p}{,} \PY{l+m+mi}{0}\PY{p}{]}\PY{p}{,} \PY{n}{P\PYZus{}int\PYZus{}affine}\PY{p}{[}\PY{l+m+mi}{1}\PY{p}{,} \PY{l+m+mi}{1}\PY{p}{]}\PY{p}{,} \PY{n}{P\PYZus{}int\PYZus{}affine}\PY{p}{[}\PY{l+m+mi}{2}\PY{p}{,} \PY{l+m+mi}{2}\PY{p}{]}\PY{p}{,} \PY{n}{P\PYZus{}int\PYZus{}affine}\PY{p}{[}\PY{l+m+mi}{2}\PY{p}{,} \PY{l+m+mi}{3}\PY{p}{]} \PY{o}{=} \PY{n}{f}\PY{o}{/}\PY{n}{z0}\PY{p}{,} \PY{n}{f}\PY{o}{/}\PY{n}{z0}\PY{p}{,} \PY{l+m+mi}{0}\PY{p}{,} \PY{l+m+mi}{1} 
             \PY{n}{P\PYZus{}int\PYZus{}affine}\PY{p}{[}\PY{l+m+mi}{0}\PY{p}{:}\PY{l+m+mi}{2}\PY{p}{,} \PY{l+m+mi}{2}\PY{p}{:}\PY{p}{]} \PY{o}{=} \PY{n}{np}\PY{o}{.}\PY{n}{array}\PY{p}{(}\PY{p}{[}\PY{p}{[}\PY{o}{\PYZhy{}}\PY{n}{f}\PY{o}{*}\PY{n}{x0} \PY{o}{/} \PY{p}{(}\PY{n}{z0}\PY{p}{)}\PY{o}{*}\PY{o}{*}\PY{l+m+mi}{2}\PY{p}{,} \PY{n}{f}\PY{o}{*}\PY{n}{x0}\PY{o}{/}\PY{n}{z0}\PY{p}{]}\PY{p}{,} \PY{p}{[}\PY{o}{\PYZhy{}}\PY{n}{f}\PY{o}{*}\PY{n}{y0} \PY{o}{/} \PY{p}{(}\PY{n}{z0}\PY{p}{)}\PY{o}{*}\PY{o}{*}\PY{l+m+mi}{2}\PY{p}{,} \PY{n}{f}\PY{o}{*}\PY{n}{y0}\PY{o}{/}\PY{n}{z0}\PY{p}{]}\PY{p}{]}\PY{p}{)}
             
             \PY{k}{return} \PY{n}{P\PYZus{}int\PYZus{}proj}\PY{p}{,} \PY{n}{P\PYZus{}int\PYZus{}affine}\PY{p}{,} \PY{n}{P\PYZus{}ext}
         
         
         \PY{c+c1}{\PYZsh{} Use the following code to display your outputs}
         \PY{c+c1}{\PYZsh{} You are free to change the axis parameters to better }
         \PY{c+c1}{\PYZsh{} display your quadrilateral but do not remove any annotations}
         
         \PY{k}{def} \PY{n+nf}{plot\PYZus{}points}\PY{p}{(}\PY{n}{points}\PY{p}{,} \PY{n}{title}\PY{o}{=}\PY{l+s+s1}{\PYZsq{}}\PY{l+s+s1}{\PYZsq{}}\PY{p}{,} \PY{n}{style}\PY{o}{=}\PY{l+s+s1}{\PYZsq{}}\PY{l+s+s1}{.\PYZhy{}r}\PY{l+s+s1}{\PYZsq{}}\PY{p}{,} \PY{n}{axis}\PY{o}{=}\PY{p}{[}\PY{p}{]}\PY{p}{)}\PY{p}{:}
             \PY{n}{inds} \PY{o}{=} \PY{n+nb}{list}\PY{p}{(}\PY{n+nb}{range}\PY{p}{(}\PY{n}{points}\PY{o}{.}\PY{n}{shape}\PY{p}{[}\PY{l+m+mi}{1}\PY{p}{]}\PY{p}{)}\PY{p}{)}\PY{o}{+}\PY{p}{[}\PY{l+m+mi}{0}\PY{p}{]}
             \PY{n}{plt}\PY{o}{.}\PY{n}{plot}\PY{p}{(}\PY{n}{points}\PY{p}{[}\PY{l+m+mi}{0}\PY{p}{,}\PY{n}{inds}\PY{p}{]}\PY{p}{,} \PY{n}{points}\PY{p}{[}\PY{l+m+mi}{1}\PY{p}{,}\PY{n}{inds}\PY{p}{]}\PY{p}{,}\PY{n}{style}\PY{p}{)}
             
             \PY{k}{for} \PY{n}{i} \PY{o+ow}{in} \PY{n+nb}{range}\PY{p}{(}\PY{n+nb}{len}\PY{p}{(}\PY{n}{points}\PY{p}{[}\PY{l+m+mi}{0}\PY{p}{,}\PY{n}{inds}\PY{p}{]}\PY{p}{)}\PY{p}{)}\PY{p}{:}
                 \PY{n}{plt}\PY{o}{.}\PY{n}{annotate}\PY{p}{(}\PY{n+nb}{str}\PY{p}{(}\PY{l+s+s2}{\PYZdq{}}\PY{l+s+s2}{\PYZob{}0:.3f\PYZcb{}}\PY{l+s+s2}{\PYZdq{}}\PY{o}{.}\PY{n}{format}\PY{p}{(}\PY{n}{points}\PY{p}{[}\PY{l+m+mi}{0}\PY{p}{,}\PY{n}{inds}\PY{p}{]}\PY{p}{[}\PY{n}{i}\PY{p}{]}\PY{p}{)}\PY{p}{)}\PY{o}{+}\PY{l+s+s2}{\PYZdq{}}\PY{l+s+s2}{,  }\PY{l+s+s2}{\PYZdq{}}\PY{o}{+}\PY{n+nb}{str}\PY{p}{(}\PY{l+s+s2}{\PYZdq{}}\PY{l+s+s2}{\PYZob{}0:.3f\PYZcb{}}\PY{l+s+s2}{\PYZdq{}}\PY{o}{.}\PY{n}{format}\PY{p}{(}\PY{n}{points}\PY{p}{[}\PY{l+m+mi}{1}\PY{p}{,}\PY{n}{inds}\PY{p}{]}\PY{p}{[}\PY{n}{i}\PY{p}{]}\PY{p}{)}\PY{p}{)}\PY{p}{,}\PY{p}{(}\PY{n}{points}\PY{p}{[}\PY{l+m+mi}{0}\PY{p}{,}\PY{n}{inds}\PY{p}{]}\PY{p}{[}\PY{n}{i}\PY{p}{]}\PY{p}{,} \PY{n}{points}\PY{p}{[}\PY{l+m+mi}{1}\PY{p}{,}\PY{n}{inds}\PY{p}{]}\PY{p}{[}\PY{n}{i}\PY{p}{]}\PY{p}{)}\PY{p}{)}
             
             \PY{k}{if} \PY{n}{title}\PY{p}{:}
                 \PY{n}{plt}\PY{o}{.}\PY{n}{title}\PY{p}{(}\PY{n}{title}\PY{p}{)}
             \PY{k}{if} \PY{n}{axis}\PY{p}{:}
                 \PY{n}{plt}\PY{o}{.}\PY{n}{axis}\PY{p}{(}\PY{n}{axis}\PY{p}{)}
                 
             \PY{n}{plt}\PY{o}{.}\PY{n}{tight\PYZus{}layout}\PY{p}{(}\PY{p}{)}
                 
         \PY{k}{def} \PY{n+nf}{main}\PY{p}{(}\PY{p}{)}\PY{p}{:}
             \PY{n}{point1} \PY{o}{=} \PY{n}{np}\PY{o}{.}\PY{n}{array}\PY{p}{(}\PY{p}{[}\PY{p}{[}\PY{o}{\PYZhy{}}\PY{l+m+mi}{1}\PY{p}{,}\PY{o}{\PYZhy{}}\PY{o}{.}\PY{l+m+mi}{5}\PY{p}{,}\PY{l+m+mi}{2}\PY{p}{]}\PY{p}{]}\PY{p}{)}\PY{o}{.}\PY{n}{T}
             \PY{n}{point2} \PY{o}{=} \PY{n}{np}\PY{o}{.}\PY{n}{array}\PY{p}{(}\PY{p}{[}\PY{p}{[}\PY{l+m+mi}{1}\PY{p}{,}\PY{o}{\PYZhy{}}\PY{o}{.}\PY{l+m+mi}{5}\PY{p}{,}\PY{l+m+mi}{2}\PY{p}{]}\PY{p}{]}\PY{p}{)}\PY{o}{.}\PY{n}{T}
             \PY{n}{point3} \PY{o}{=} \PY{n}{np}\PY{o}{.}\PY{n}{array}\PY{p}{(}\PY{p}{[}\PY{p}{[}\PY{l+m+mi}{1}\PY{p}{,}\PY{o}{.}\PY{l+m+mi}{5}\PY{p}{,}\PY{l+m+mi}{2}\PY{p}{]}\PY{p}{]}\PY{p}{)}\PY{o}{.}\PY{n}{T}
             \PY{n}{point4} \PY{o}{=} \PY{n}{np}\PY{o}{.}\PY{n}{array}\PY{p}{(}\PY{p}{[}\PY{p}{[}\PY{o}{\PYZhy{}}\PY{l+m+mi}{1}\PY{p}{,}\PY{o}{.}\PY{l+m+mi}{5}\PY{p}{,}\PY{l+m+mi}{2}\PY{p}{]}\PY{p}{]}\PY{p}{)}\PY{o}{.}\PY{n}{T}
             \PY{n}{points} \PY{o}{=} \PY{n}{np}\PY{o}{.}\PY{n}{hstack}\PY{p}{(}\PY{p}{(}\PY{n}{point1}\PY{p}{,}\PY{n}{point2}\PY{p}{,}\PY{n}{point3}\PY{p}{,}\PY{n}{point4}\PY{p}{)}\PY{p}{)}
             \PY{n}{axis} \PY{o}{=} \PY{p}{[}\PY{p}{[}\PY{o}{\PYZhy{}}\PY{o}{.}\PY{l+m+mi}{6}\PY{p}{,} \PY{o}{.}\PY{l+m+mi}{6}\PY{p}{,}\PY{o}{\PYZhy{}}\PY{o}{.}\PY{l+m+mi}{6}\PY{p}{,}\PY{o}{.}\PY{l+m+mi}{6}\PY{p}{]}\PY{p}{,}\PY{p}{[}\PY{o}{\PYZhy{}}\PY{o}{.}\PY{l+m+mi}{6}\PY{p}{,} \PY{o}{.}\PY{l+m+mi}{6}\PY{p}{,}\PY{o}{\PYZhy{}}\PY{o}{.}\PY{l+m+mi}{6}\PY{p}{,}\PY{o}{.}\PY{l+m+mi}{6}\PY{p}{]}\PY{p}{,}\PY{p}{[}\PY{l+m+mi}{0}\PY{p}{,} \PY{l+m+mf}{2.5}\PY{p}{,}\PY{o}{\PYZhy{}}\PY{o}{.}\PY{l+m+mi}{6}\PY{p}{,}\PY{o}{.}\PY{l+m+mi}{7}\PY{p}{]}\PY{p}{,}\PY{p}{[}\PY{l+m+mi}{0}\PY{p}{,} \PY{l+m+mi}{1}\PY{p}{,}\PY{o}{\PYZhy{}}\PY{o}{.}\PY{l+m+mi}{6}\PY{p}{,}\PY{o}{.}\PY{l+m+mi}{6}\PY{p}{]}\PY{p}{]}
             \PY{k}{for} \PY{n}{i}\PY{p}{,} \PY{n}{camera} \PY{o+ow}{in} \PY{n+nb}{enumerate}\PY{p}{(}\PY{p}{[}\PY{n}{camera1}\PY{p}{,} \PY{n}{camera2}\PY{p}{,} \PY{n}{camera3}\PY{p}{,} \PY{n}{camera4}\PY{p}{]}\PY{p}{)}\PY{p}{:}
                 \PY{n}{P\PYZus{}int\PYZus{}proj}\PY{p}{,} \PY{n}{P\PYZus{}int\PYZus{}affine}\PY{p}{,} \PY{n}{P\PYZus{}ext} \PY{o}{=} \PY{n}{camera}\PY{p}{(}\PY{p}{)}
                 
                 \PY{n}{plt}\PY{o}{.}\PY{n}{subplot}\PY{p}{(}\PY{l+m+mi}{1}\PY{p}{,} \PY{l+m+mi}{2}\PY{p}{,} \PY{l+m+mi}{1}\PY{p}{)}
                 \PY{n}{plot\PYZus{}points}\PY{p}{(}\PY{n}{project\PYZus{}points}\PY{p}{(}\PY{n}{P\PYZus{}int\PYZus{}proj}\PY{p}{,} \PY{n}{P\PYZus{}ext}\PY{p}{,} \PY{n}{points}\PY{p}{)}\PY{p}{,} \PY{n}{title}\PY{o}{=}\PY{l+s+s1}{\PYZsq{}}\PY{l+s+s1}{Camera }\PY{l+s+si}{\PYZpc{}d}\PY{l+s+s1}{ Projective}\PY{l+s+s1}{\PYZsq{}}\PY{o}{\PYZpc{}}\PY{p}{(}\PY{n}{i}\PY{o}{+}\PY{l+m+mi}{1}\PY{p}{)}\PY{p}{,} \PY{n}{axis} \PY{o}{=} \PY{n}{axis}\PY{p}{[}\PY{n}{i}\PY{p}{]}\PY{p}{)}
                 
                 \PY{n}{plt}\PY{o}{.}\PY{n}{subplot}\PY{p}{(}\PY{l+m+mi}{1}\PY{p}{,} \PY{l+m+mi}{2}\PY{p}{,} \PY{l+m+mi}{2}\PY{p}{)}
                 \PY{n}{plot\PYZus{}points}\PY{p}{(}\PY{n}{project\PYZus{}points}\PY{p}{(}\PY{n}{P\PYZus{}int\PYZus{}affine}\PY{p}{,} \PY{n}{P\PYZus{}ext}\PY{p}{,} \PY{n}{points}\PY{p}{)}\PY{p}{,} \PY{n}{title}\PY{o}{=}\PY{l+s+s1}{\PYZsq{}}\PY{l+s+s1}{Camera }\PY{l+s+si}{\PYZpc{}d}\PY{l+s+s1}{ Affine}\PY{l+s+s1}{\PYZsq{}}\PY{o}{\PYZpc{}}\PY{p}{(}\PY{n}{i}\PY{o}{+}\PY{l+m+mi}{1}\PY{p}{)}\PY{p}{,} \PY{n}{axis} \PY{o}{=} \PY{n}{axis}\PY{p}{[}\PY{n}{i}\PY{p}{]}\PY{p}{)}
                 
                 \PY{n}{plt}\PY{o}{.}\PY{n}{show}\PY{p}{(}\PY{p}{)}
         
                 
         \PY{n}{main}\PY{p}{(}\PY{p}{)}
\end{Verbatim}


    \begin{center}
    \adjustimage{max size={0.9\linewidth}{0.9\paperheight}}{output_5_0.png}
    \end{center}
    { \hspace*{\fill} \\}
    
    \begin{center}
    \adjustimage{max size={0.9\linewidth}{0.9\paperheight}}{output_5_1.png}
    \end{center}
    { \hspace*{\fill} \\}
    
    \begin{center}
    \adjustimage{max size={0.9\linewidth}{0.9\paperheight}}{output_5_2.png}
    \end{center}
    { \hspace*{\fill} \\}
    
    \begin{center}
    \adjustimage{max size={0.9\linewidth}{0.9\paperheight}}{output_5_3.png}
    \end{center}
    { \hspace*{\fill} \\}
    
    \#\#\# Written answers: Actual points for affine camera model for each
camera: i choose P = {[}0, 0, 2{]} in world coordinate system as the
original point. But before points being photographed with intrinsic
camera matrix, they need to be transformed with extrinsic matrix. So,
the actual points for affine camera model are camera 1: P' = {[}0, 0,
2{]}; camera 2: P' = {[}0, 0, 3{]}; camera 3: P' = {[}1.732, 0, 2{]};
camera 4: P' = {[}1.732, 0, 14{]}

Affine projection is an approximation of perspective projection. So, in
most cases, it is sililar to perspective projection. The results for
camera 1 and camear 2 show that there is no difference between these two
projection since there are no rotations. Affine projection preserves
parallism. All the results for affine camera model show this property.
For camera 3, the translation is small for the local length and rotation
matrix. The points with affine projection looks very differnt from those
with perspective projection. The values of focal length, rotation matrix
and translation matrix affect the performation of affine projection.

    \subsection{Problem 5: Homography {[}12
pts{]}}\label{problem-5-homography-12-pts}

You may use eig or svd routines in python for this part of the
assignment.

Consider a vision application in which components of the scene are
replaced by components from another image scene.

In this problem, we will implement partial functionality of a smartphone
camera scanning application (Example: CamScanner) that, in case you've
never used before, takes pictures of documents and transforms it by
warping and aligning to give an image similar to one which would've been
obtained through using a scanner.

The transformation can be visualized by imagining the use of two cameras
forming an image of a scene with a document. The scene would be the
document you're trying to scan placed on a table and one of the cameras
would be your smart phone camera, forming the image that you'll be
uploading and using in this assignment. There can also be an ideally
placed camera, oriented in the world in such a way that the image it
forms of the scene has the document perfectly algined. While it is
unlikely you can hold your phone still enough to get such an image, we
can use homography to transform the image you take into the image that
the ideally placed camera would have taken.

This digital replacement is accomplished by a set of corresponding
points for the document in both the source (your picture) and target
(the ideal) images. The task then consists of mapping the points from
the source to their respective points in the target image. In the most
general case, there would be no constraints on the scene geometry,
making the problem quite hard to solve. If, however, the scene can be
approximated by a plane in 3D, a solution can be formulated much more
easily even without the knowledge of camera calibration parameters.

To solve this section of the homework, you will begin by understanding
the transformation that maps one image onto another in the planar scene
case. Then you will write a program that implements this transformation
and use it to warp some document into a well aligned document (See the
given example to understand what we mean by well aligned).

To begin with, we consider the projection of planes in images. imagine
two cameras \(C_1\) and \(C_2\) looking at a plane \(\pi\) in the world.
Consider a point \(P\) on the plane \(\pi\) and its projection
\(p=[\text{u1, v1, 1}]^T\) in the image 1 and \(q=[\text{u2, v2, 1}]^T\)
in image 2.

There exists a unique, upto scale, 3 \(\times\) 3 matrix \(H\) such
that, for any point \(P\): \[q \approx Hp\] Here \(\approx\) denotes
equality in homogeneous coordinates, meaning that the left and right
hand sides are proportional. Note that \(H\) only depends on the plane
and the projection matrices of the two cameras.

The interesting thing about this result is that by using \(H\) we can
compute the image of \(P\) that would be seen in the camera with center
\(C_2\) from the image of the point in the camera with center at
\(C_1\), without knowing the three dimensional location. Such an \(H\)
is a projective transformation of the plane, called a homography.

In this problem, complete the code for computeH and warp functions that
can be used in the skeletal code that follows.

There are three warp functions to implement in this assignment, example
ouputs of which are shown below. In warp1, you will create a homography
from points in your image to the target image (Mapping source points to
target points). In warp2, the inverse of this process will be done. In
warp3, you will create a homography between a given image and your
image, replacing your document with the given image.

\begin{enumerate}
\def\labelenumi{\arabic{enumi}.}
\item
   2. 3.
\item
  In the context of this problem, the source image refers to the image
  of a document you take that needs to be replaced into the target.
\item
  The target image can start out as an empty matrix that you fill out
  using your code.
\item
  You will have to implement the computeH function that computes a
  homography. It takes in the point correspondences between the source
  image and target image in homogeneous coordinates respectively and
  returns a 3 \(\times\) 3 homography matrix.
\item
  You will also have to implement the three warp functions in the
  skeleton code given and plot the resultant image pairs. For plotting,
  make sure that the target image is not smaller than the source image.
\end{enumerate}

Note: We have provided test code to check if your implementation for
computeH is correct. All the code to plot the results needed is also
provided along with the code to read in the images and other data
required for this problem. Please try not to modify that code.

You may find following python built-ins helpful: numpy.linalg.svd,
numpy.meshgrid

    \begin{Verbatim}[commandchars=\\\{\}]
{\color{incolor}In [{\color{incolor}43}]:} \PY{k+kn}{import} \PY{n+nn}{numpy} \PY{k+kn}{as} \PY{n+nn}{np}
         \PY{k+kn}{from} \PY{n+nn}{scipy.misc} \PY{k+kn}{import} \PY{n}{imread}\PY{p}{,} \PY{n}{imresize}
         \PY{k+kn}{from} \PY{n+nn}{scipy.io} \PY{k+kn}{import} \PY{n}{loadmat}
         \PY{k+kn}{import} \PY{n+nn}{matplotlib.pyplot} \PY{k+kn}{as} \PY{n+nn}{plt}
         
         \PY{c+c1}{\PYZsh{} load image to be used \PYZhy{} resize to make sure it\PYZsq{}s not too large}
         \PY{c+c1}{\PYZsh{} You can use the given image as well}
         \PY{c+c1}{\PYZsh{} A large image will make testing you code take longer; once you\PYZsq{}re satisfied with your result,}
         \PY{c+c1}{\PYZsh{} you can, if you wish to, make the image larger (or till your computer memory allows you to)}
         \PY{n}{source\PYZus{}image} \PY{o}{=} \PY{n}{imresize}\PY{p}{(}\PY{n}{imread}\PY{p}{(}\PY{l+s+s2}{\PYZdq{}}\PY{l+s+s2}{photo.jpg}\PY{l+s+s2}{\PYZdq{}}\PY{p}{)}\PY{p}{,}\PY{o}{.}\PY{l+m+mi}{1}\PY{p}{)}\PY{p}{[}\PY{p}{:}\PY{p}{,}\PY{p}{:}\PY{p}{,}\PY{p}{:}\PY{l+m+mi}{3}\PY{p}{]}\PY{o}{/}\PY{l+m+mf}{255.}
         
         \PY{c+c1}{\PYZsh{} display images}
         \PY{n}{plt}\PY{o}{.}\PY{n}{imshow}\PY{p}{(}\PY{n}{source\PYZus{}image}\PY{p}{)}
         
         \PY{c+c1}{\PYZsh{} Align the polygon such that the corners align with the document in your picture}
         \PY{c+c1}{\PYZsh{} This polygon doesn\PYZsq{}t need to overlap with the edges perftectly, an approximation is fine}
         \PY{c+c1}{\PYZsh{} The order of points is clockwise, starting from bottom left.}
         \PY{n}{x\PYZus{}coords} \PY{o}{=} \PY{p}{[}\PY{l+m+mi}{11}\PY{p}{,}\PY{l+m+mi}{50}\PY{p}{,}\PY{l+m+mi}{195}\PY{p}{,}\PY{l+m+mi}{195}\PY{p}{]} 
         \PY{n}{y\PYZus{}coords} \PY{o}{=} \PY{p}{[}\PY{l+m+mi}{115}\PY{p}{,}\PY{l+m+mi}{10}\PY{p}{,}\PY{l+m+mi}{19}\PY{p}{,}\PY{l+m+mi}{150}\PY{p}{]}
         
         \PY{c+c1}{\PYZsh{} Plot points from the previous problem is used to draw over your image }
         \PY{c+c1}{\PYZsh{} Note that your coordinates will change once you resize your image again}
         \PY{n}{source\PYZus{}points} \PY{o}{=} \PY{n}{np}\PY{o}{.}\PY{n}{vstack}\PY{p}{(}\PY{p}{(}\PY{n}{x\PYZus{}coords}\PY{p}{,} \PY{n}{y\PYZus{}coords}\PY{p}{)}\PY{p}{)}
         \PY{n}{plot\PYZus{}points}\PY{p}{(}\PY{n}{source\PYZus{}points}\PY{p}{)}
         
         \PY{n}{plt}\PY{o}{.}\PY{n}{show}\PY{p}{(}\PY{p}{)}
\end{Verbatim}


    \begin{Verbatim}[commandchars=\\\{\}]
/Users/huangzhisheng/anaconda2/lib/python2.7/site-packages/ipykernel\_launcher.py:10: DeprecationWarning: `imread` is deprecated!
`imread` is deprecated in SciPy 1.0.0, and will be removed in 1.2.0.
Use ``imageio.imread`` instead.
  \# Remove the CWD from sys.path while we load stuff.
/Users/huangzhisheng/anaconda2/lib/python2.7/site-packages/ipykernel\_launcher.py:10: DeprecationWarning: `imresize` is deprecated!
`imresize` is deprecated in SciPy 1.0.0, and will be removed in 1.2.0.
Use ``skimage.transform.resize`` instead.
  \# Remove the CWD from sys.path while we load stuff.

    \end{Verbatim}

    \begin{center}
    \adjustimage{max size={0.9\linewidth}{0.9\paperheight}}{output_8_1.png}
    \end{center}
    { \hspace*{\fill} \\}
    
    \begin{Verbatim}[commandchars=\\\{\}]
{\color{incolor}In [{\color{incolor}44}]:} \PY{k}{def} \PY{n+nf}{computeH}\PY{p}{(}\PY{n}{source\PYZus{}points}\PY{p}{,} \PY{n}{target\PYZus{}points}\PY{p}{)}\PY{p}{:}
             \PY{c+c1}{\PYZsh{} returns the 3x3 homography matrix such that:}
             \PY{c+c1}{\PYZsh{} np.matmul(H, source\PYZus{}points) \PYZti{} target\PYZus{}points}
             \PY{c+c1}{\PYZsh{} where source\PYZus{}points and target\PYZus{}points are expected to be in homogeneous}
             
             \PY{c+c1}{\PYZsh{} Please refer the note on DLT algorithm given at: }
             \PY{c+c1}{\PYZsh{} https://cseweb.ucsd.edu/classes/wi07/cse252a/homography\PYZus{}estimation/homography\PYZus{}estimation.pdf}
             
             \PY{c+c1}{\PYZsh{} make sure points are 3D homogeneous}
             \PY{k}{assert} \PY{n}{source\PYZus{}points}\PY{o}{.}\PY{n}{shape}\PY{p}{[}\PY{l+m+mi}{0}\PY{p}{]}\PY{o}{==}\PY{l+m+mi}{3} \PY{o+ow}{and} \PY{n}{target\PYZus{}points}\PY{o}{.}\PY{n}{shape}\PY{p}{[}\PY{l+m+mi}{0}\PY{p}{]}\PY{o}{==}\PY{l+m+mi}{3}
             
             \PY{n}{x1}\PY{p}{,} \PY{n}{y1}\PY{p}{,} \PY{n}{z1} \PY{o}{=} \PY{n}{source\PYZus{}points}\PY{p}{[}\PY{l+m+mi}{0}\PY{p}{,} \PY{p}{:}\PY{p}{]}\PY{p}{,}\PY{n}{source\PYZus{}points}\PY{p}{[}\PY{l+m+mi}{1}\PY{p}{,} \PY{p}{:}\PY{p}{]}\PY{p}{,}\PY{n}{source\PYZus{}points}\PY{p}{[}\PY{l+m+mi}{2}\PY{p}{,} \PY{p}{:}\PY{p}{]} \PY{c+c1}{\PYZsh{} get x1, y1,z1 from source points}
             \PY{n}{x2}\PY{p}{,} \PY{n}{y2}\PY{p}{,} \PY{n}{z2} \PY{o}{=} \PY{n}{target\PYZus{}points}\PY{p}{[}\PY{l+m+mi}{0}\PY{p}{,} \PY{p}{:}\PY{p}{]}\PY{p}{,}\PY{n}{target\PYZus{}points}\PY{p}{[}\PY{l+m+mi}{1}\PY{p}{,} \PY{p}{:}\PY{p}{]}\PY{p}{,}\PY{n}{target\PYZus{}points}\PY{p}{[}\PY{l+m+mi}{2}\PY{p}{,} \PY{p}{:}\PY{p}{]} \PY{c+c1}{\PYZsh{} get x2, y2, z2 from target points}
             \PY{n}{x2\PYZus{}}\PY{p}{,} \PY{n}{y2\PYZus{}} \PY{o}{=} \PY{n}{x2} \PY{o}{/} \PY{n}{z2}\PY{p}{,} \PY{n}{y2} \PY{o}{/} \PY{n}{z2} \PY{c+c1}{\PYZsh{} generate x2\PYZsq{} and y2\PYZsq{}}
             \PY{n}{m} \PY{o}{=} \PY{n}{source\PYZus{}points}\PY{o}{.}\PY{n}{shape}\PY{p}{[}\PY{l+m+mi}{1}\PY{p}{]} \PY{c+c1}{\PYZsh{} get the number of points}
         
             \PY{n}{A1} \PY{o}{=} \PY{n}{np}\PY{o}{.}\PY{n}{vstack}\PY{p}{(}\PY{p}{(}\PY{o}{\PYZhy{}}\PY{n}{x1}\PY{p}{,} \PY{o}{\PYZhy{}}\PY{n}{y1}\PY{p}{,} \PY{o}{\PYZhy{}}\PY{l+m+mi}{1}\PY{o}{*}\PY{n}{np}\PY{o}{.}\PY{n}{ones}\PY{p}{(}\PY{n}{m}\PY{p}{)}\PY{p}{,} \PY{n}{np}\PY{o}{.}\PY{n}{zeros}\PY{p}{(}\PY{n}{m}\PY{p}{)}\PY{p}{,}\PY{n}{np}\PY{o}{.}\PY{n}{zeros}\PY{p}{(}\PY{n}{m}\PY{p}{)}\PY{p}{,}\PY{n}{np}\PY{o}{.}\PY{n}{zeros}\PY{p}{(}\PY{n}{m}\PY{p}{)}\PY{p}{,} \PY{n}{x2\PYZus{}}\PY{o}{*}\PY{n}{x1}\PY{p}{,} \PY{n}{x2\PYZus{}}\PY{o}{*}\PY{n}{y1}\PY{p}{,} \PY{n}{x2\PYZus{}}\PY{p}{)}\PY{p}{)}\PY{o}{.}\PY{n}{T}
             \PY{c+c1}{\PYZsh{}print(A1)}
             \PY{n}{A2} \PY{o}{=} \PY{n}{np}\PY{o}{.}\PY{n}{vstack}\PY{p}{(}\PY{p}{(}\PY{n}{np}\PY{o}{.}\PY{n}{zeros}\PY{p}{(}\PY{n}{m}\PY{p}{)}\PY{p}{,} \PY{n}{np}\PY{o}{.}\PY{n}{zeros}\PY{p}{(}\PY{n}{m}\PY{p}{)}\PY{p}{,} \PY{n}{np}\PY{o}{.}\PY{n}{zeros}\PY{p}{(}\PY{n}{m}\PY{p}{)}\PY{p}{,} \PY{o}{\PYZhy{}}\PY{l+m+mi}{1}\PY{o}{*}\PY{n}{x1}\PY{p}{,} \PY{o}{\PYZhy{}}\PY{l+m+mi}{1}\PY{o}{*}\PY{n}{y1}\PY{p}{,} \PY{o}{\PYZhy{}}\PY{l+m+mi}{1}\PY{o}{*}\PY{n}{np}\PY{o}{.}\PY{n}{ones}\PY{p}{(}\PY{n}{m}\PY{p}{)}\PY{p}{,} \PY{n}{y2\PYZus{}}\PY{o}{*}\PY{n}{x1}\PY{p}{,}\PY{n}{y2\PYZus{}}\PY{o}{*}\PY{n}{y1}\PY{p}{,} \PY{n}{y2\PYZus{}}\PY{p}{)}\PY{p}{)}\PY{o}{.}\PY{n}{T}
             \PY{c+c1}{\PYZsh{}print(A2)}
             \PY{n}{A} \PY{o}{=} \PY{n}{np}\PY{o}{.}\PY{n}{vstack}\PY{p}{(}\PY{p}{(}\PY{n}{A1}\PY{p}{,} \PY{n}{A2}\PY{p}{)}\PY{p}{)} \PY{c+c1}{\PYZsh{} generate A matrix to perform SVD on}
             \PY{c+c1}{\PYZsh{}print(A)}
             \PY{n}{u}\PY{p}{,} \PY{n}{s}\PY{p}{,} \PY{n}{vh} \PY{o}{=} \PY{n}{np}\PY{o}{.}\PY{n}{linalg}\PY{o}{.}\PY{n}{svd}\PY{p}{(}\PY{n}{A}\PY{p}{)} \PY{c+c1}{\PYZsh{} perform SVD on A matrix}
             \PY{n}{H} \PY{o}{=} \PY{n}{vh}\PY{o}{.}\PY{n}{T}\PY{p}{[}\PY{p}{:}\PY{p}{,}\PY{o}{\PYZhy{}}\PY{l+m+mi}{1}\PY{p}{]} \PY{c+c1}{\PYZsh{} get the eigenvector for sigma 9}
             \PY{k}{return} \PY{n}{H}\PY{o}{.}\PY{n}{reshape}\PY{p}{(}\PY{p}{(}\PY{l+m+mi}{3}\PY{p}{,} \PY{l+m+mi}{3}\PY{p}{)}\PY{p}{)} \PY{c+c1}{\PYZsh{} reshape the eigenvector to 3x3 matrix}
             
             
         
         \PY{c+c1}{\PYZsh{}\PYZsh{}\PYZsh{}\PYZsh{}\PYZsh{}\PYZsh{}\PYZsh{}\PYZsh{}\PYZsh{}\PYZsh{}\PYZsh{}\PYZsh{}\PYZsh{}\PYZsh{}\PYZsh{}\PYZsh{}\PYZsh{}\PYZsh{}\PYZsh{}\PYZsh{}\PYZsh{}\PYZsh{}\PYZsh{}\PYZsh{}\PYZsh{}\PYZsh{}\PYZsh{}\PYZsh{}\PYZsh{}\PYZsh{}\PYZsh{}\PYZsh{}\PYZsh{}\PYZsh{}\PYZsh{}\PYZsh{}\PYZsh{}\PYZsh{}\PYZsh{}\PYZsh{}\PYZsh{}\PYZsh{}\PYZsh{}\PYZsh{}\PYZsh{}\PYZsh{}\PYZsh{}\PYZsh{}\PYZsh{}\PYZsh{}\PYZsh{}\PYZsh{}\PYZsh{}\PYZsh{}\PYZsh{}}
         \PY{c+c1}{\PYZsh{} test code. Do not modify}
         \PY{c+c1}{\PYZsh{}\PYZsh{}\PYZsh{}\PYZsh{}\PYZsh{}\PYZsh{}\PYZsh{}\PYZsh{}\PYZsh{}\PYZsh{}\PYZsh{}\PYZsh{}\PYZsh{}\PYZsh{}\PYZsh{}\PYZsh{}\PYZsh{}\PYZsh{}\PYZsh{}\PYZsh{}\PYZsh{}\PYZsh{}\PYZsh{}\PYZsh{}\PYZsh{}\PYZsh{}\PYZsh{}\PYZsh{}\PYZsh{}\PYZsh{}\PYZsh{}\PYZsh{}\PYZsh{}\PYZsh{}\PYZsh{}\PYZsh{}\PYZsh{}\PYZsh{}\PYZsh{}\PYZsh{}\PYZsh{}\PYZsh{}\PYZsh{}\PYZsh{}\PYZsh{}\PYZsh{}\PYZsh{}\PYZsh{}\PYZsh{}\PYZsh{}\PYZsh{}\PYZsh{}\PYZsh{}\PYZsh{}\PYZsh{}}
         \PY{k}{def} \PY{n+nf}{test\PYZus{}computeH}\PY{p}{(}\PY{p}{)}\PY{p}{:}
             \PY{n}{source\PYZus{}points} \PY{o}{=} \PY{n}{np}\PY{o}{.}\PY{n}{array}\PY{p}{(}\PY{p}{[}\PY{p}{[}\PY{l+m+mi}{0}\PY{p}{,}\PY{l+m+mf}{0.5}\PY{p}{]}\PY{p}{,}\PY{p}{[}\PY{l+m+mi}{1}\PY{p}{,}\PY{l+m+mf}{0.5}\PY{p}{]}\PY{p}{,}\PY{p}{[}\PY{l+m+mi}{1}\PY{p}{,}\PY{l+m+mf}{1.5}\PY{p}{]}\PY{p}{,}\PY{p}{[}\PY{l+m+mi}{0}\PY{p}{,}\PY{l+m+mf}{1.5}\PY{p}{]}\PY{p}{]}\PY{p}{)}\PY{o}{.}\PY{n}{T}
             \PY{n}{target\PYZus{}points} \PY{o}{=} \PY{n}{np}\PY{o}{.}\PY{n}{array}\PY{p}{(}\PY{p}{[}\PY{p}{[}\PY{l+m+mi}{0}\PY{p}{,}\PY{l+m+mi}{0}\PY{p}{]}\PY{p}{,}\PY{p}{[}\PY{l+m+mi}{1}\PY{p}{,}\PY{l+m+mi}{0}\PY{p}{]}\PY{p}{,}\PY{p}{[}\PY{l+m+mi}{2}\PY{p}{,}\PY{l+m+mi}{1}\PY{p}{]}\PY{p}{,}\PY{p}{[}\PY{o}{\PYZhy{}}\PY{l+m+mi}{1}\PY{p}{,}\PY{l+m+mi}{1}\PY{p}{]}\PY{p}{]}\PY{p}{)}\PY{o}{.}\PY{n}{T}
             \PY{n}{H} \PY{o}{=} \PY{n}{computeH}\PY{p}{(}\PY{n}{to\PYZus{}homog}\PY{p}{(}\PY{n}{source\PYZus{}points}\PY{p}{)}\PY{p}{,} \PY{n}{to\PYZus{}homog}\PY{p}{(}\PY{n}{target\PYZus{}points}\PY{p}{)}\PY{p}{)}
             \PY{n}{mapped\PYZus{}points} \PY{o}{=} \PY{n}{from\PYZus{}homog}\PY{p}{(}\PY{n}{np}\PY{o}{.}\PY{n}{matmul}\PY{p}{(}\PY{n}{H}\PY{p}{,}\PY{n}{to\PYZus{}homog}\PY{p}{(}\PY{n}{source\PYZus{}points}\PY{p}{)}\PY{p}{)}\PY{p}{)}
             \PY{n}{plot\PYZus{}points}\PY{p}{(}\PY{n}{source\PYZus{}points}\PY{p}{,}\PY{n}{style}\PY{o}{=}\PY{l+s+s1}{\PYZsq{}}\PY{l+s+s1}{.\PYZhy{}k}\PY{l+s+s1}{\PYZsq{}}\PY{p}{)}
             \PY{n}{plot\PYZus{}points}\PY{p}{(}\PY{n}{target\PYZus{}points}\PY{p}{,}\PY{n}{style}\PY{o}{=}\PY{l+s+s1}{\PYZsq{}}\PY{l+s+s1}{*\PYZhy{}b}\PY{l+s+s1}{\PYZsq{}}\PY{p}{)}
             \PY{n}{plot\PYZus{}points}\PY{p}{(}\PY{n}{mapped\PYZus{}points}\PY{p}{,}\PY{n}{style}\PY{o}{=}\PY{l+s+s1}{\PYZsq{}}\PY{l+s+s1}{.:r}\PY{l+s+s1}{\PYZsq{}}\PY{p}{)}
             \PY{n}{plt}\PY{o}{.}\PY{n}{show}\PY{p}{(}\PY{p}{)}
             \PY{k}{print}\PY{p}{(}\PY{l+s+s1}{\PYZsq{}}\PY{l+s+s1}{The red and blue quadrilaterals should overlap if ComputeH is implemented correctly.}\PY{l+s+s1}{\PYZsq{}}\PY{p}{)}
         \PY{n}{test\PYZus{}computeH}\PY{p}{(}\PY{p}{)}
\end{Verbatim}


    \begin{center}
    \adjustimage{max size={0.9\linewidth}{0.9\paperheight}}{output_9_0.png}
    \end{center}
    { \hspace*{\fill} \\}
    
    \begin{Verbatim}[commandchars=\\\{\}]
The red and blue quadrilaterals should overlap if ComputeH is implemented correctly.

    \end{Verbatim}

    \begin{Verbatim}[commandchars=\\\{\}]
{\color{incolor}In [{\color{incolor}45}]:} \PY{k}{def} \PY{n+nf}{warp}\PY{p}{(}\PY{n}{source\PYZus{}img}\PY{p}{,} \PY{n}{source\PYZus{}points}\PY{p}{,} \PY{n}{target\PYZus{}size}\PY{p}{)}\PY{p}{:}
             \PY{c+c1}{\PYZsh{} Create a target image and select target points to create a homography from source image to target image,}
             \PY{c+c1}{\PYZsh{} in other words map all source points to target points and then create}
             \PY{c+c1}{\PYZsh{} a warped version of the image based on the homography by filling in the target image.}
             \PY{c+c1}{\PYZsh{} Make sure the new image (of size target\PYZus{}size) has the same number of color channels as source image}
             \PY{k}{assert} \PY{n}{target\PYZus{}size}\PY{p}{[}\PY{l+m+mi}{2}\PY{p}{]}\PY{o}{==}\PY{n}{source\PYZus{}img}\PY{o}{.}\PY{n}{shape}\PY{p}{[}\PY{l+m+mi}{2}\PY{p}{]}
             \PY{n}{m} \PY{o}{=} \PY{n}{target\PYZus{}size}\PY{p}{[}\PY{l+m+mi}{0}\PY{p}{]}
             \PY{n}{n} \PY{o}{=} \PY{n}{target\PYZus{}size}\PY{p}{[}\PY{l+m+mi}{1}\PY{p}{]}
             \PY{n}{target\PYZus{}points} \PY{o}{=} \PY{n}{np}\PY{o}{.}\PY{n}{array}\PY{p}{(}\PY{p}{[}\PY{p}{[}\PY{l+m+mi}{0}\PY{p}{,}\PY{l+m+mi}{0}\PY{p}{]}\PY{p}{,}\PY{p}{[}\PY{n}{n}\PY{o}{\PYZhy{}}\PY{l+m+mi}{1}\PY{p}{,} \PY{l+m+mi}{0}\PY{p}{]}\PY{p}{,}\PY{p}{[}\PY{n}{n}\PY{o}{\PYZhy{}}\PY{l+m+mi}{1}\PY{p}{,} \PY{n}{m}\PY{o}{\PYZhy{}}\PY{l+m+mi}{1}\PY{p}{]}\PY{p}{,}\PY{p}{[}\PY{l+m+mi}{0}\PY{p}{,} \PY{n}{m} \PY{o}{\PYZhy{}} \PY{l+m+mi}{1}\PY{p}{]}\PY{p}{]}\PY{p}{)}\PY{o}{.}\PY{n}{T} \PY{c+c1}{\PYZsh{} get the target\PYZus{}points}
             
             \PY{n}{H} \PY{o}{=} \PY{n}{computeH}\PY{p}{(}\PY{n}{to\PYZus{}homog}\PY{p}{(}\PY{n}{source\PYZus{}points}\PY{p}{)}\PY{p}{,} \PY{n}{to\PYZus{}homog}\PY{p}{(}\PY{n}{target\PYZus{}points}\PY{p}{)}\PY{p}{)} \PY{c+c1}{\PYZsh{} generate the homegraphy matrix}
             
             \PY{n}{x} \PY{o}{=} \PY{n}{np}\PY{o}{.}\PY{n}{linspace}\PY{p}{(}\PY{l+m+mi}{0}\PY{p}{,} \PY{n}{source\PYZus{}img}\PY{o}{.}\PY{n}{shape}\PY{p}{[}\PY{l+m+mi}{1}\PY{p}{]} \PY{o}{\PYZhy{}} \PY{l+m+mi}{1}\PY{p}{,} \PY{n}{source\PYZus{}img}\PY{o}{.}\PY{n}{shape}\PY{p}{[}\PY{l+m+mi}{1}\PY{p}{]}\PY{p}{)}
             \PY{n}{y} \PY{o}{=} \PY{n}{np}\PY{o}{.}\PY{n}{linspace}\PY{p}{(}\PY{l+m+mi}{0}\PY{p}{,} \PY{n}{source\PYZus{}img}\PY{o}{.}\PY{n}{shape}\PY{p}{[}\PY{l+m+mi}{0}\PY{p}{]} \PY{o}{\PYZhy{}} \PY{l+m+mi}{1}\PY{p}{,} \PY{n}{source\PYZus{}img}\PY{o}{.}\PY{n}{shape}\PY{p}{[}\PY{l+m+mi}{0}\PY{p}{]}\PY{p}{)}
             \PY{n}{xv1}\PY{p}{,} \PY{n}{yv1} \PY{o}{=} \PY{n}{np}\PY{o}{.}\PY{n}{meshgrid}\PY{p}{(}\PY{n}{x}\PY{p}{,} \PY{n}{y}\PY{p}{)}
             \PY{n}{xv} \PY{o}{=} \PY{n}{xv1}\PY{o}{.}\PY{n}{reshape}\PY{p}{(}\PY{p}{(}\PY{l+m+mi}{1}\PY{p}{,}\PY{o}{\PYZhy{}}\PY{l+m+mi}{1}\PY{p}{)}\PY{p}{)}
             \PY{n}{yv} \PY{o}{=} \PY{n}{yv1}\PY{o}{.}\PY{n}{reshape}\PY{p}{(}\PY{p}{(}\PY{l+m+mi}{1}\PY{p}{,} \PY{o}{\PYZhy{}}\PY{l+m+mi}{1}\PY{p}{)}\PY{p}{)}
         
             \PY{n}{sourceMatrix} \PY{o}{=} \PY{n}{np}\PY{o}{.}\PY{n}{vstack}\PY{p}{(}\PY{p}{(}\PY{n}{xv}\PY{p}{,} \PY{n}{yv}\PY{p}{)}\PY{p}{)}
             \PY{n}{mappedPoints} \PY{o}{=} \PY{n}{from\PYZus{}homog}\PY{p}{(}\PY{n}{np}\PY{o}{.}\PY{n}{matmul}\PY{p}{(}\PY{n}{H}\PY{p}{,}\PY{n}{to\PYZus{}homog}\PY{p}{(}\PY{n}{sourceMatrix}\PY{p}{)}\PY{p}{)}\PY{p}{)} \PY{c+c1}{\PYZsh{} mapping from source to target}
             \PY{n}{xNew} \PY{o}{=} \PY{n}{mappedPoints}\PY{p}{[}\PY{l+m+mi}{0}\PY{p}{,} \PY{p}{:}\PY{p}{]}\PY{o}{.}\PY{n}{reshape}\PY{p}{(}\PY{n}{source\PYZus{}img}\PY{o}{.}\PY{n}{shape}\PY{p}{[}\PY{l+m+mi}{0}\PY{p}{]}\PY{p}{,} \PY{o}{\PYZhy{}}\PY{l+m+mi}{1}\PY{p}{)}
             \PY{n}{yNew} \PY{o}{=} \PY{n}{mappedPoints}\PY{p}{[}\PY{l+m+mi}{1}\PY{p}{,} \PY{p}{:}\PY{p}{]}\PY{o}{.}\PY{n}{reshape}\PY{p}{(}\PY{n}{source\PYZus{}img}\PY{o}{.}\PY{n}{shape}\PY{p}{[}\PY{l+m+mi}{0}\PY{p}{]}\PY{p}{,} \PY{o}{\PYZhy{}}\PY{l+m+mi}{1}\PY{p}{)}
             
             \PY{n}{targetImg} \PY{o}{=} \PY{n}{np}\PY{o}{.}\PY{n}{ones}\PY{p}{(}\PY{p}{(}\PY{n}{m}\PY{p}{,} \PY{n}{n}\PY{p}{,} \PY{l+m+mi}{3}\PY{p}{)}\PY{p}{)} \PY{c+c1}{\PYZsh{} create a target image with all\PYZhy{}white pixels}
         
             \PY{c+c1}{\PYZsh{} filling in the target image}
             \PY{k}{for} \PY{n}{i} \PY{o+ow}{in} \PY{n+nb}{xrange}\PY{p}{(}\PY{n}{source\PYZus{}img}\PY{o}{.}\PY{n}{shape}\PY{p}{[}\PY{l+m+mi}{0}\PY{p}{]}\PY{p}{)}\PY{p}{:}
                 \PY{k}{for} \PY{n}{j} \PY{o+ow}{in} \PY{n+nb}{xrange}\PY{p}{(}\PY{n}{source\PYZus{}img}\PY{o}{.}\PY{n}{shape}\PY{p}{[}\PY{l+m+mi}{1}\PY{p}{]}\PY{p}{)}\PY{p}{:}
                     \PY{k}{if} \PY{l+m+mi}{0} \PY{o}{\PYZlt{}}\PY{o}{=} \PY{n}{np}\PY{o}{.}\PY{n}{ceil}\PY{p}{(}\PY{n}{xNew}\PY{p}{[}\PY{n}{i}\PY{p}{,} \PY{n}{j}\PY{p}{]}\PY{p}{)} \PY{o}{\PYZlt{}}\PY{o}{=} \PY{n}{n} \PY{o}{\PYZhy{}} \PY{l+m+mi}{1}  \PY{o+ow}{and} \PY{l+m+mi}{0} \PY{o}{\PYZlt{}}\PY{o}{=} \PY{n}{np}\PY{o}{.}\PY{n}{ceil}\PY{p}{(}\PY{n}{yNew}\PY{p}{[}\PY{n}{i}\PY{p}{,} \PY{n}{j}\PY{p}{]}\PY{p}{)} \PY{o}{\PYZlt{}}\PY{o}{=} \PY{n}{m} \PY{o}{\PYZhy{}} \PY{l+m+mi}{1} \PY{p}{:}
                         \PY{n}{targetImg}\PY{p}{[}\PY{n}{np}\PY{o}{.}\PY{n}{int}\PY{p}{(}\PY{n}{np}\PY{o}{.}\PY{n}{ceil}\PY{p}{(}\PY{n}{yNew}\PY{p}{[}\PY{n}{i}\PY{p}{,} \PY{n}{j}\PY{p}{]}\PY{p}{)}\PY{p}{)}\PY{p}{,} \PY{n}{np}\PY{o}{.}\PY{n}{int}\PY{p}{(}\PY{n}{np}\PY{o}{.}\PY{n}{ceil}\PY{p}{(}\PY{n}{xNew}\PY{p}{[}\PY{n}{i}\PY{p}{,} \PY{n}{j}\PY{p}{]}\PY{p}{)}\PY{p}{)}\PY{p}{,} \PY{p}{:}\PY{p}{]} \PY{o}{=} \PY{n}{source\PYZus{}img}\PY{p}{[}\PY{n}{i}\PY{p}{,} \PY{n}{j}\PY{p}{,} \PY{p}{:}\PY{p}{]}
             
             \PY{k}{return} \PY{n}{targetImg}
             
             
             
             
         
         \PY{c+c1}{\PYZsh{} Use the code below to plot your result}
         \PY{n}{result} \PY{o}{=} \PY{n}{warp}\PY{p}{(}\PY{n}{source\PYZus{}image}\PY{p}{,} \PY{n}{source\PYZus{}points}\PY{p}{,} \PY{p}{(}\PY{l+m+mi}{200}\PY{p}{,}\PY{l+m+mi}{140}\PY{p}{,}\PY{l+m+mi}{3}\PY{p}{)}\PY{p}{)}
         \PY{n}{plt}\PY{o}{.}\PY{n}{subplot}\PY{p}{(}\PY{l+m+mi}{1}\PY{p}{,} \PY{l+m+mi}{2}\PY{p}{,} \PY{l+m+mi}{1}\PY{p}{)}
         \PY{n}{plt}\PY{o}{.}\PY{n}{imshow}\PY{p}{(}\PY{n}{source\PYZus{}image}\PY{p}{)}
         \PY{n}{plt}\PY{o}{.}\PY{n}{subplot}\PY{p}{(}\PY{l+m+mi}{1}\PY{p}{,} \PY{l+m+mi}{2}\PY{p}{,} \PY{l+m+mi}{2}\PY{p}{)}
         \PY{n}{plt}\PY{o}{.}\PY{n}{imshow}\PY{p}{(}\PY{n}{result}\PY{p}{)}
         \PY{n}{plt}\PY{o}{.}\PY{n}{show}\PY{p}{(}\PY{p}{)}
\end{Verbatim}


    \begin{center}
    \adjustimage{max size={0.9\linewidth}{0.9\paperheight}}{output_10_0.png}
    \end{center}
    { \hspace*{\fill} \\}
    
    The output of warp1 of your code probably has some striations or noise.
The larger you make your target image, the less it will resemble the
document in the source image. Why is this happening?

To fix this, implement warp2, by creating an inverse homography matrix
and fill in the target image.

    \begin{Verbatim}[commandchars=\\\{\}]
{\color{incolor}In [{\color{incolor}46}]:} \PY{k}{def} \PY{n+nf}{warp2}\PY{p}{(}\PY{n}{source\PYZus{}img}\PY{p}{,} \PY{n}{source\PYZus{}points}\PY{p}{,} \PY{n}{target\PYZus{}size}\PY{p}{)}\PY{p}{:}
             \PY{c+c1}{\PYZsh{} Create a target image and select target points to create a homography from target image to source image,}
             \PY{c+c1}{\PYZsh{} in other words map each target point to a source point, and then create a warped version}
             \PY{c+c1}{\PYZsh{} of the image based on the homography by filling in the target image.}
             \PY{c+c1}{\PYZsh{} Make sure the new image (of size target\PYZus{}size) has the same number of color channels as source image}
             \PY{k}{assert} \PY{n}{target\PYZus{}size}\PY{p}{[}\PY{l+m+mi}{2}\PY{p}{]} \PY{o}{==} \PY{n}{source\PYZus{}img}\PY{o}{.}\PY{n}{shape}\PY{p}{[}\PY{l+m+mi}{2}\PY{p}{]}
             \PY{n}{m} \PY{o}{=} \PY{n}{target\PYZus{}size}\PY{p}{[}\PY{l+m+mi}{0}\PY{p}{]}
             \PY{n}{n} \PY{o}{=} \PY{n}{target\PYZus{}size}\PY{p}{[}\PY{l+m+mi}{1}\PY{p}{]}
             \PY{n}{target\PYZus{}points} \PY{o}{=} \PY{n}{np}\PY{o}{.}\PY{n}{array}\PY{p}{(}\PY{p}{[}\PY{p}{[}\PY{l+m+mi}{0}\PY{p}{,}\PY{l+m+mi}{0}\PY{p}{]}\PY{p}{,}\PY{p}{[}\PY{n}{n}\PY{o}{\PYZhy{}}\PY{l+m+mi}{1}\PY{p}{,} \PY{l+m+mi}{0}\PY{p}{]}\PY{p}{,}\PY{p}{[}\PY{n}{n}\PY{o}{\PYZhy{}}\PY{l+m+mi}{1}\PY{p}{,} \PY{n}{m}\PY{o}{\PYZhy{}}\PY{l+m+mi}{1}\PY{p}{]}\PY{p}{,}\PY{p}{[}\PY{l+m+mi}{0}\PY{p}{,} \PY{n}{m} \PY{o}{\PYZhy{}} \PY{l+m+mi}{1}\PY{p}{]}\PY{p}{]}\PY{p}{)}\PY{o}{.}\PY{n}{T} \PY{c+c1}{\PYZsh{} get the target points}
             \PY{n}{targetImg} \PY{o}{=} \PY{n}{np}\PY{o}{.}\PY{n}{ones}\PY{p}{(}\PY{p}{(}\PY{n}{m}\PY{p}{,} \PY{n}{n}\PY{p}{,} \PY{l+m+mi}{3}\PY{p}{)}\PY{p}{)}
             \PY{n}{sourceImg} \PY{o}{=} \PY{n}{source\PYZus{}image}
         
             \PY{n}{x} \PY{o}{=} \PY{n}{np}\PY{o}{.}\PY{n}{linspace}\PY{p}{(}\PY{l+m+mi}{0}\PY{p}{,} \PY{n}{n} \PY{o}{\PYZhy{}} \PY{l+m+mi}{1}\PY{p}{,} \PY{n}{n}\PY{p}{)}
             \PY{n}{y} \PY{o}{=} \PY{n}{np}\PY{o}{.}\PY{n}{linspace}\PY{p}{(}\PY{l+m+mi}{0}\PY{p}{,} \PY{n}{m} \PY{o}{\PYZhy{}} \PY{l+m+mi}{1}\PY{p}{,} \PY{n}{m}\PY{p}{)}
             \PY{n}{xv1}\PY{p}{,} \PY{n}{yv1} \PY{o}{=} \PY{n}{np}\PY{o}{.}\PY{n}{meshgrid}\PY{p}{(}\PY{n}{x}\PY{p}{,} \PY{n}{y}\PY{p}{)}
         
             \PY{n}{H} \PY{o}{=} \PY{n}{computeH}\PY{p}{(}\PY{n}{to\PYZus{}homog}\PY{p}{(}\PY{n}{target\PYZus{}points}\PY{p}{)}\PY{p}{,} \PY{n}{to\PYZus{}homog}\PY{p}{(}\PY{n}{source\PYZus{}points}\PY{p}{)}\PY{p}{)} \PY{c+c1}{\PYZsh{} generate the homegrapy matrix for target to source}
         
             \PY{n}{xv} \PY{o}{=} \PY{n}{xv1}\PY{o}{.}\PY{n}{reshape}\PY{p}{(}\PY{p}{(}\PY{l+m+mi}{1}\PY{p}{,}\PY{o}{\PYZhy{}}\PY{l+m+mi}{1}\PY{p}{)}\PY{p}{)}
             \PY{n}{yv} \PY{o}{=} \PY{n}{yv1}\PY{o}{.}\PY{n}{reshape}\PY{p}{(}\PY{p}{(}\PY{l+m+mi}{1}\PY{p}{,} \PY{o}{\PYZhy{}}\PY{l+m+mi}{1}\PY{p}{)}\PY{p}{)}
         
             \PY{n}{targetMatrix} \PY{o}{=} \PY{n}{np}\PY{o}{.}\PY{n}{vstack}\PY{p}{(}\PY{p}{(}\PY{n}{xv}\PY{p}{,} \PY{n}{yv}\PY{p}{)}\PY{p}{)}
             \PY{n}{mappedPoints} \PY{o}{=} \PY{n}{from\PYZus{}homog}\PY{p}{(}\PY{n}{np}\PY{o}{.}\PY{n}{matmul}\PY{p}{(}\PY{n}{H}\PY{p}{,} \PY{n}{to\PYZus{}homog}\PY{p}{(}\PY{n}{targetMatrix}\PY{p}{)}\PY{p}{)}\PY{p}{)} \PY{c+c1}{\PYZsh{} generate the transformed coordinates}
             \PY{n}{xNew} \PY{o}{=} \PY{n}{mappedPoints}\PY{p}{[}\PY{l+m+mi}{0}\PY{p}{,} \PY{p}{:}\PY{p}{]}\PY{o}{.}\PY{n}{reshape}\PY{p}{(}\PY{n}{m}\PY{p}{,} \PY{o}{\PYZhy{}}\PY{l+m+mi}{1}\PY{p}{)}
             \PY{n}{yNew} \PY{o}{=} \PY{n}{mappedPoints}\PY{p}{[}\PY{l+m+mi}{1}\PY{p}{,} \PY{p}{:}\PY{p}{]}\PY{o}{.}\PY{n}{reshape}\PY{p}{(}\PY{n}{m}\PY{p}{,} \PY{o}{\PYZhy{}}\PY{l+m+mi}{1}\PY{p}{)}
             
             \PY{c+c1}{\PYZsh{} filling the target image}
             \PY{k}{for} \PY{n}{i} \PY{o+ow}{in} \PY{n+nb}{xrange}\PY{p}{(}\PY{n}{m}\PY{p}{)}\PY{p}{:}
                 \PY{k}{for} \PY{n}{j} \PY{o+ow}{in} \PY{n+nb}{xrange}\PY{p}{(}\PY{n}{n}\PY{p}{)}\PY{p}{:}
                     \PY{n}{targetImg}\PY{p}{[}\PY{n}{i}\PY{p}{,}\PY{n}{j}\PY{p}{,} \PY{p}{:}\PY{p}{]} \PY{o}{=} \PY{n}{source\PYZus{}image}\PY{p}{[}\PY{n}{np}\PY{o}{.}\PY{n}{int}\PY{p}{(}\PY{n}{np}\PY{o}{.}\PY{n}{ceil}\PY{p}{(}\PY{n}{yNew}\PY{p}{[}\PY{n}{i}\PY{p}{,} \PY{n}{j}\PY{p}{]}\PY{p}{)}\PY{p}{)}\PY{p}{,} \PY{n}{np}\PY{o}{.}\PY{n}{int}\PY{p}{(}\PY{n}{np}\PY{o}{.}\PY{n}{ceil}\PY{p}{(}\PY{n}{xNew}\PY{p}{[}\PY{n}{i}\PY{p}{,} \PY{n}{j}\PY{p}{]}\PY{p}{)}\PY{p}{)}\PY{p}{,} \PY{p}{:}\PY{p}{]}
                 
             \PY{k}{return} \PY{n}{targetImg}
         
         
         \PY{c+c1}{\PYZsh{} Use the code below to plot your result}
         \PY{n}{size\PYZus{}factor} \PY{o}{=} \PY{l+m+mi}{2}
         \PY{n}{result} \PY{o}{=} \PY{n}{warp2}\PY{p}{(}\PY{n}{source\PYZus{}image}\PY{p}{,} \PY{n}{source\PYZus{}points}\PY{p}{,} \PY{p}{(}\PY{l+m+mi}{600}\PY{o}{*}\PY{n}{size\PYZus{}factor}\PY{p}{,}\PY{l+m+mi}{420}\PY{o}{*}\PY{n}{size\PYZus{}factor}\PY{p}{,}\PY{l+m+mi}{3}\PY{p}{)}\PY{p}{)}
         \PY{n}{plt}\PY{o}{.}\PY{n}{subplot}\PY{p}{(}\PY{l+m+mi}{1}\PY{p}{,} \PY{l+m+mi}{2}\PY{p}{,} \PY{l+m+mi}{1}\PY{p}{)}
         \PY{n}{plt}\PY{o}{.}\PY{n}{imshow}\PY{p}{(}\PY{n}{source\PYZus{}image}\PY{p}{)}
         \PY{n}{plt}\PY{o}{.}\PY{n}{subplot}\PY{p}{(}\PY{l+m+mi}{1}\PY{p}{,} \PY{l+m+mi}{2}\PY{p}{,} \PY{l+m+mi}{2}\PY{p}{)}
         \PY{n}{plt}\PY{o}{.}\PY{n}{imshow}\PY{p}{(}\PY{n}{result}\PY{p}{)}
         \PY{n}{plt}\PY{o}{.}\PY{n}{show}\PY{p}{(}\PY{p}{)}
\end{Verbatim}


    \begin{center}
    \adjustimage{max size={0.9\linewidth}{0.9\paperheight}}{output_12_0.png}
    \end{center}
    { \hspace*{\fill} \\}
    
    Try playing around with the size of your target image in warp1 versus in
warp2, additionally you can also implement nearest pixel interpolation
or bi-linear interpolations and see if that makes a difference in your
output.

In warp3, you'll be replacing the document in your image with a provided
image. Read in "ucsd\_logo.png" as the source image, keeping your
document as the target.

    \begin{Verbatim}[commandchars=\\\{\}]
{\color{incolor}In [{\color{incolor}47}]:} \PY{c+c1}{\PYZsh{} Load the given UCSD logo image}
         \PY{n}{source\PYZus{}image2} \PY{o}{=} \PY{n}{imread}\PY{p}{(}\PY{l+s+s1}{\PYZsq{}}\PY{l+s+s1}{ucsd\PYZus{}logo.png}\PY{l+s+s1}{\PYZsq{}}\PY{p}{)}\PY{p}{[}\PY{p}{:}\PY{p}{,}\PY{p}{:}\PY{p}{,}\PY{p}{:}\PY{l+m+mi}{3}\PY{p}{]}\PY{o}{/}\PY{l+m+mf}{255.}
         
         \PY{k}{def} \PY{n+nf}{warp3}\PY{p}{(}\PY{n}{target\PYZus{}image}\PY{p}{,} \PY{n}{target\PYZus{}points}\PY{p}{,} \PY{n}{source\PYZus{}image}\PY{p}{)}\PY{p}{:}
             
             \PY{n}{m} \PY{o}{=} \PY{n}{source\PYZus{}image}\PY{o}{.}\PY{n}{shape}\PY{p}{[}\PY{l+m+mi}{0}\PY{p}{]}
             \PY{n}{n} \PY{o}{=} \PY{n}{source\PYZus{}image}\PY{o}{.}\PY{n}{shape}\PY{p}{[}\PY{l+m+mi}{1}\PY{p}{]}
             \PY{n}{source\PYZus{}points} \PY{o}{=} \PY{n}{np}\PY{o}{.}\PY{n}{array}\PY{p}{(}\PY{p}{[}\PY{p}{[}\PY{l+m+mi}{0}\PY{p}{,}\PY{l+m+mi}{0}\PY{p}{]}\PY{p}{,}\PY{p}{[}\PY{l+m+mi}{0}\PY{p}{,}\PY{n}{n}\PY{o}{\PYZhy{}}\PY{l+m+mi}{1}\PY{p}{]}\PY{p}{,}\PY{p}{[}\PY{n}{m}\PY{o}{\PYZhy{}}\PY{l+m+mi}{1}\PY{p}{,} \PY{n}{n}\PY{o}{\PYZhy{}}\PY{l+m+mi}{1}\PY{p}{]}\PY{p}{,}\PY{p}{[}\PY{n}{m}\PY{o}{\PYZhy{}}\PY{l+m+mi}{1}\PY{p}{,} \PY{l+m+mi}{0}\PY{p}{]}\PY{p}{]}\PY{p}{)}\PY{o}{.}\PY{n}{T} \PY{c+c1}{\PYZsh{} get the points of source}
             
             
             \PY{n}{sourceImg} \PY{o}{=} \PY{n}{source\PYZus{}image}
             \PY{n}{x} \PY{o}{=} \PY{n}{np}\PY{o}{.}\PY{n}{linspace}\PY{p}{(}\PY{l+m+mi}{0}\PY{p}{,} \PY{n}{sourceImg}\PY{o}{.}\PY{n}{shape}\PY{p}{[}\PY{l+m+mi}{1}\PY{p}{]}\PY{o}{\PYZhy{}}\PY{l+m+mi}{1}\PY{p}{,} \PY{n}{sourceImg}\PY{o}{.}\PY{n}{shape}\PY{p}{[}\PY{l+m+mi}{1}\PY{p}{]}\PY{p}{)}
             \PY{n}{y} \PY{o}{=} \PY{n}{np}\PY{o}{.}\PY{n}{linspace}\PY{p}{(}\PY{l+m+mi}{0}\PY{p}{,} \PY{n}{sourceImg}\PY{o}{.}\PY{n}{shape}\PY{p}{[}\PY{l+m+mi}{0}\PY{p}{]}\PY{o}{\PYZhy{}}\PY{l+m+mi}{1}\PY{p}{,} \PY{n}{sourceImg}\PY{o}{.}\PY{n}{shape}\PY{p}{[}\PY{l+m+mi}{0}\PY{p}{]}\PY{p}{)}
             \PY{n}{xv1}\PY{p}{,} \PY{n}{yv1} \PY{o}{=} \PY{n}{np}\PY{o}{.}\PY{n}{meshgrid}\PY{p}{(}\PY{n}{y}\PY{p}{,} \PY{n}{x}\PY{p}{)}
         
         
             \PY{n}{H} \PY{o}{=} \PY{n}{computeH}\PY{p}{(}\PY{n}{to\PYZus{}homog}\PY{p}{(}\PY{n}{source\PYZus{}points}\PY{p}{)}\PY{p}{,} \PY{n}{to\PYZus{}homog}\PY{p}{(}\PY{n}{target\PYZus{}points}\PY{p}{)}\PY{p}{)} \PY{c+c1}{\PYZsh{} generate the homegrapy matrix from source to target}
         
             \PY{n}{xv} \PY{o}{=} \PY{n}{xv1}\PY{o}{.}\PY{n}{reshape}\PY{p}{(}\PY{p}{(}\PY{l+m+mi}{1}\PY{p}{,}\PY{o}{\PYZhy{}}\PY{l+m+mi}{1}\PY{p}{)}\PY{p}{)}
             \PY{n}{yv} \PY{o}{=} \PY{n}{yv1}\PY{o}{.}\PY{n}{reshape}\PY{p}{(}\PY{p}{(}\PY{l+m+mi}{1}\PY{p}{,} \PY{o}{\PYZhy{}}\PY{l+m+mi}{1}\PY{p}{)}\PY{p}{)}
             
             \PY{n}{sourceMatrix} \PY{o}{=} \PY{n}{np}\PY{o}{.}\PY{n}{vstack}\PY{p}{(}\PY{p}{(}\PY{n}{xv}\PY{p}{,} \PY{n}{yv}\PY{p}{)}\PY{p}{)}
             \PY{n}{mappedPoints} \PY{o}{=} \PY{n}{from\PYZus{}homog}\PY{p}{(}\PY{n}{np}\PY{o}{.}\PY{n}{matmul}\PY{p}{(}\PY{n}{H}\PY{p}{,} \PY{n}{to\PYZus{}homog}\PY{p}{(}\PY{n}{sourceMatrix}\PY{p}{)}\PY{p}{)}\PY{p}{)} \PY{c+c1}{\PYZsh{} get the transformed coordinated from source to target}
             \PY{n}{xNew} \PY{o}{=} \PY{n}{mappedPoints}\PY{p}{[}\PY{l+m+mi}{0}\PY{p}{,} \PY{p}{:}\PY{p}{]}\PY{o}{.}\PY{n}{reshape}\PY{p}{(}\PY{n}{sourceImg}\PY{o}{.}\PY{n}{shape}\PY{p}{[}\PY{l+m+mi}{0}\PY{p}{]}\PY{p}{,} \PY{o}{\PYZhy{}}\PY{l+m+mi}{1}\PY{p}{)}
             \PY{n}{yNew} \PY{o}{=} \PY{n}{mappedPoints}\PY{p}{[}\PY{l+m+mi}{1}\PY{p}{,} \PY{p}{:}\PY{p}{]}\PY{o}{.}\PY{n}{reshape}\PY{p}{(}\PY{n}{sourceImg}\PY{o}{.}\PY{n}{shape}\PY{p}{[}\PY{l+m+mi}{0}\PY{p}{]}\PY{p}{,} \PY{o}{\PYZhy{}}\PY{l+m+mi}{1}\PY{p}{)}
             
             \PY{c+c1}{\PYZsh{} filling the desire parts}
             \PY{k}{for} \PY{n}{i} \PY{o+ow}{in} \PY{n+nb}{xrange}\PY{p}{(}\PY{n}{sourceImg}\PY{o}{.}\PY{n}{shape}\PY{p}{[}\PY{l+m+mi}{0}\PY{p}{]}\PY{p}{)}\PY{p}{:}
                 \PY{k}{for} \PY{n}{j} \PY{o+ow}{in} \PY{n+nb}{xrange}\PY{p}{(}\PY{n}{sourceImg}\PY{o}{.}\PY{n}{shape}\PY{p}{[}\PY{l+m+mi}{1}\PY{p}{]}\PY{p}{)}\PY{p}{:}
                    \PY{c+c1}{\PYZsh{} if 19\PYZlt{}= np.ceil(yNew[i, j]) \PYZlt{}= 195 and 10 \PYZlt{}= np.ceil(xNew[i, j]) \PYZlt{}= 150:}
                     \PY{n}{target\PYZus{}image}\PY{p}{[}\PY{n}{np}\PY{o}{.}\PY{n}{int}\PY{p}{(}\PY{n}{np}\PY{o}{.}\PY{n}{ceil}\PY{p}{(}\PY{n}{yNew}\PY{p}{[}\PY{n}{i}\PY{p}{,} \PY{n}{j}\PY{p}{]}\PY{p}{)}\PY{p}{)}\PY{p}{,} \PY{n}{np}\PY{o}{.}\PY{n}{int}\PY{p}{(}\PY{n}{np}\PY{o}{.}\PY{n}{ceil}\PY{p}{(}\PY{n}{xNew}\PY{p}{[}\PY{n}{i}\PY{p}{,} \PY{n}{j}\PY{p}{]}\PY{p}{)}\PY{p}{)}\PY{p}{,} \PY{p}{:}\PY{p}{]} \PY{o}{=} \PY{n}{sourceImg}\PY{p}{[}\PY{n}{j}\PY{p}{,} \PY{n}{i}\PY{p}{,} \PY{p}{:}\PY{p}{]}
                     \PY{c+c1}{\PYZsh{}source\PYZus{}image[j, i, :]}
             \PY{k}{return} \PY{n}{target\PYZus{}image}
         
         
         \PY{c+c1}{\PYZsh{} Use the code below to plot your result}
         \PY{n}{result} \PY{o}{=} \PY{n}{warp3}\PY{p}{(}\PY{n}{source\PYZus{}image}\PY{p}{,} \PY{n}{source\PYZus{}points}\PY{p}{,} \PY{n}{source\PYZus{}image2}\PY{p}{)}
         \PY{n}{plt}\PY{o}{.}\PY{n}{subplot}\PY{p}{(}\PY{l+m+mi}{1}\PY{p}{,} \PY{l+m+mi}{2}\PY{p}{,} \PY{l+m+mi}{1}\PY{p}{)}
         \PY{n}{plt}\PY{o}{.}\PY{n}{imshow}\PY{p}{(}\PY{n}{source\PYZus{}image2}\PY{p}{)}
         \PY{n}{plt}\PY{o}{.}\PY{n}{subplot}\PY{p}{(}\PY{l+m+mi}{1}\PY{p}{,} \PY{l+m+mi}{2}\PY{p}{,} \PY{l+m+mi}{2}\PY{p}{)}
         \PY{n}{plt}\PY{o}{.}\PY{n}{imshow}\PY{p}{(}\PY{n}{result}\PY{p}{)}
         \PY{n}{plt}\PY{o}{.}\PY{n}{show}\PY{p}{(}\PY{p}{)}
\end{Verbatim}


    \begin{Verbatim}[commandchars=\\\{\}]
/Users/huangzhisheng/anaconda2/lib/python2.7/site-packages/ipykernel\_launcher.py:2: DeprecationWarning: `imread` is deprecated!
`imread` is deprecated in SciPy 1.0.0, and will be removed in 1.2.0.
Use ``imageio.imread`` instead.
  

    \end{Verbatim}

    \begin{center}
    \adjustimage{max size={0.9\linewidth}{0.9\paperheight}}{output_14_1.png}
    \end{center}
    { \hspace*{\fill} \\}
    

    % Add a bibliography block to the postdoc
    
    
    
    \end{document}
